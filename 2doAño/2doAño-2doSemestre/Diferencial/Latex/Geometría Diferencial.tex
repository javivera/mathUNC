\documentclass[12pt]{book}

\usepackage{amssymb}
\usepackage{amsmath}
\usepackage{amsthm}
\usepackage{mathpazo}
\usepackage{tcolorbox}
\usepackage[margin=0.8in]{geometry}
\usepackage[colorlinks=true]{hyperref}
\usepackage{tcolorbox}
\usepackage{bookmark}
\usepackage{titlesec}
\usepackage{helvet}

\titleformat{\chapter}
  {\normalfont\fontsize{16}{19}\sffamily\bfseries}
  {\thechapter}
  {1em}
  {}

\titleformat{\section}
  {\normalfont\fontsize{12}{17}\sffamily\bfseries}
  {\thesection}
  {1em}
  {}

\titleformat{\subsection}
  {\normalfont\fontsize{12}{17}\sffamily\bfseries\slshape}
  {\thesubsection}
  {1em}
  {}

\newtheoremstyle{break}
  {\topsep}{\topsep}%
  {\itshape}{}%
  {\bfseries}{}%
  {\newline}{}%
\theoremstyle{break}
\newtheorem{theorem}{Teorema}[section]
\newtheorem{corollary}{Corolario}[theorem]
\newtheorem{lemma}[theorem]{Lema}
\newtheorem{proposition}{Proposición}
\newtheorem*{remark}{Observación}
\newtheorem{definition}{Definición}[section]


\begin{document}
    % LaTeX

\title{\textsc{\footnotesize Geometría Diferencial} \hfill
\includegraphics[scale=0.15]{logo_UNC-FAMAF.png} \\ 
Resumen}
\author{Javier Vera}
\maketitle
\newpage

\section{Superficies Regulares}

\begin{definition}
  Un subconjunto $S \subseteq \mathbb{R}^3 $ es una superficie regular si para cada $p \in S$ existe un entorno
  $V$ en $\mathbb{R}^3 $, un abierto $U \subseteq \mathbb{R}^2 $ y un mapa $\phi:U\rightarrow V\cap S$ que cumple
  \begin{enumerate}
      \item $\phi$ es differenciable si lo miramos como $\phi:U\rightarrow \mathbb{R}^3 $. 
      Esto significa que si escribimos
      $$ \phi(u,v) = (\phi_1(u,v),\phi_2(u,v),\phi_2(u,v)) \quad (u,v) \in U$$
      Entonces $\phi_1(u,v),\phi_2(u,v),\phi_3(u,v)$ tienen derivadas parciales de todos los órdenes
      en $U$
      \item $\phi$ es un homeomorfismo, es decir $\phi$ tiene inversa $\phi^{-1}:V\cap S \rightarrow U$
      que es contínua, esto es $\phi^{-1}$ es una restrición de una mapa contínuo 
      $F:W \subseteq \mathbb{R}^3 \rightarrow \mathbb{R}^2$ definido en un abierto $W$ que contiene a $V\cap S$
      \item Para cada $q\in U$, la differencial $d\phi_q:\mathbb{R}^2 \rightarrow \mathbb{R}^3$ es inyectivo
  \end{enumerate}
\end{definition}

\section{Diferenciabilidad}
\begin{definition}
    Sea $f:V\rightarrow \mathbb{R}^n$ una función definida en un abierto $V$ de una superficie regular $S$.
    Decimos que $f$ es diferenciable en $p\in V$ si existe alguna parametrización 
    $\phi:U \subseteq \mathbb{R}^2 \rightarrow S$ con 
    $p\in \phi(U) \subseteq V$ tal que $f\circ \phi:U \subseteq \mathbb{R}^2 \rightarrow \mathbb{R}^n$ es
    diferenciable en $\phi^{-1}(p)$. 
    
    Además decimos que $f$ es diferenciable en $V$ si lo és para todo $p \in V$
\end{definition}

\begin{definition}
    Sea $f:V \subseteq \mathbb{R}^n\rightarrow S$ con $V$ abierto se dice que $f$ es diferenciable en $p\in V$
    si existe $\phi:U\rightarrow S\cap V$ parametrización de $f(p) \in \phi(U)$, $\tilde{V} \subseteq V$,
    $\tilde{V}$ entorno abierto de $p$ tal que $f(\tilde{V}) \subseteq \phi(U)$ y 
    $\phi^{-1}\circ f: \mathbb{R}^n \rightarrow U$ es
    dif en $p$. 

    Además decimos que $f$ es diferenciable en $V$ si $f$ lo es $\forall p \in V$
\end{definition}

\begin{definition}
    Sean $S_1,S_2$ superficies regulares $f:S_1\rightarrow S_2$ y $p\in S_1$. Decimos que $f$ es 
    diferenciable en $p$ si existen parametrizaciones $\phi:U\rightarrow S_1$ y $\psi:V\rightarrow S_2$
    de $p$ y $f(p)$ respectivamente tales que 
    \begin{enumerate}
        \item $f(\phi(U)) \subseteq \psi(V)$
        \item $\psi^{-1}\circ f \circ \phi $ es diferenciable en $\phi^{-1}(p)$
    \end{enumerate}
    $f$ se dice diferenciable en $S_1$ si lo es para todo $p \in S_1$  
\end{definition}

\begin{definition}
    Dos superficies reguleres $S_1$ y $S_2$ se dicen difeomorfas si existe una biyección $f:S_1\rightarrow S_2$
    diferenciable con inversa diferenciable (homemorfismo). Una tal $f$ se llama difeomorfismo
\end{definition}

\begin{definition}
    Localmente diferenciable p1 clase 11
\end{definition}

\section{Orientabilidad}

\begin{proposition}
  Sea $V \subseteq \mathbb{R}^3 $ abierto, $S$ superficie. clase 9,10
\end{proposition}

\begin{proposition}
    Dado $R \subseteq S$ región acotada contenida en $\phi(U)$ para cierta carta $\phi:U\rightarrow S$. Se 
    define el área de $R$ de la siguiente manera: 
    $$A(R) = \int \int_{Q=\phi^{-1}(R)} ||\phi_u(u,v) \times \phi_v(u,v)||dudv$$
\end{proposition}

\begin{definition}
    página 1 clase 12 orientable
\end{definition}

\begin{proposition}
    Sea $S$ una superficie regular supongamos que existe una $\phi:U\rightarrow S$ y $\psi:V\rightarrow S$
    parametrizaciones tales que $\phi(U)\cup \psi(V) = S$ y $\phi(U)\cap \psi(V) = W$ es conexo entonces 
    $S$ es orientable
\end{proposition}

\section{Isometrías}
\begin{proposition}
  Dada $F:S_1\rightarrow S_2$ se dice que $F$ es una isometría si es un difeomorfismo tal que
  $$<dF_p(v),dF_p(w)>_{F(p)} = <v,w>_p \quad \forall p\in S_1 \quad \forall v,w \in T_pS_1$$

  Esta condición es equivalente a $$ I_{F(p)}(dF_p(v)) = I_p(v) \quad \forall p \in S_1 , v \in T_pS_1$$
\end{proposition}

\begin{proposition}
  Localemnte isometrica clase 17 p8    
\end{proposition}

\section{Egregium}
\begin{proposition}
    Los simbolos de cristoffer son las coordenadas de phi uu , phi uv , phi vv en base phiu , phiv,N 
    pagina clase 18 
\end{proposition}

\end{document}
