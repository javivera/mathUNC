\documentclass{article}

\usepackage[english]{babel}
\usepackage[utf8]{inputenc}
\usepackage{amsmath,amssymb}
\usepackage{amsthm}
\usepackage{parskip}
\usepackage{graphicx}

% Margins
\usepackage[top=2.5cm, left=3cm, right=3cm, bottom=4.0cm]{geometry}
% Colour table cells
\usepackage[table]{xcolor}

% Get larger line spacing in table
\newcommand{\tablespace}{\\[1.25mm]}
\newcommand\Tstrut{\rule{0pt}{2.6ex}}         % = `top' strut
\newcommand\tstrut{\rule{0pt}{2.0ex}}         % = `top' strut
\newcommand\Bstrut{\rule[-0.9ex]{0pt}{0pt}}   % = `bottom' strut

%%%%%%%%%%%%%%%%%
%     Title     %
%%%%%%%%%%%%%%%%%
\title{Coursework template CO343}
\author{Firstname Lastname \\ CID 01234567}
\date{\today}

\begin{document}
\maketitle

%%%%%%%%%%%%%%%%%
%   Problem 1   %
%%%%%%%%%%%%%%%%%
\section{Problem 1}
    Definamos una base \[\mathcal{B}= \{v_1,\ldots,v_n\} \]

    Sabemos que $T(v_j) = 1.v_j$ para $i=1\ldots n$ por lo tanto
    \[[T]_{ \mathcal{B} } = 
    \begin{bmatrix}
        \vert & & \vert \\
        e_1   & \cdots & e_n   \\
        \vert & & \vert
    \end{bmatrix}
    \]

    Por lo tanto es directo notar que \[X_c = (x-1)^n\]
    Usando Cayley-Hamilton $X_m = (x-1)^k$ con $k\leq n$

    Pero evaluando la matriz en $x-1$ nos da 0 , por lo tanto es el 
    pol minimal

    Usando las mismas ideas en el operador nulo vemos que
     $X_c = x^n$ y $X_m=x$

\section{Problema 2}
    \[ \mathcal{X}_a= det
    \begin{pmatrix}
    x+9 & -4 & -4\\
    8 & x-3 & -4 \\
    16 & -8 & x-7 \\
    \end{pmatrix} 
    = x^3 - x^2 -5x -3 = (x+1)^2 (x-3)\]

    La base de auto vectores seria \[\]
\end{document}