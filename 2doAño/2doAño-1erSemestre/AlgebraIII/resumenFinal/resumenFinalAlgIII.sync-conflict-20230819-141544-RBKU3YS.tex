\documentclass{article}

\usepackage{amssymb}
\usepackage{amsmath}
\usepackage{amsthm}
\usepackage{mathpazo}
\usepackage{tcolorbox}
\usepackage[margin=0.8in]{geometry}
\usepackage[colorlinks=true]{hyperref}
\usepackage{tcolorbox}
\usepackage{bookmark}

\newtheoremstyle{break}
  {\topsep}{\topsep}%
  {\itshape}{}%
  {\bfseries}{}%
  {\newline}{}%
\theoremstyle{break}
\newtheorem{theorem}{Teorema}[section]
\newtheorem{corollary}{Corolario}[theorem]
\newtheorem{lemma}[theorem]{Lema}
\newtheorem{proposition}{Proposición}
\newtheorem*{remark}{Observación}
\newtheorem{definition}{Definición}[section]

\begin{document}
    % LaTeX
    
\title{Resumen Final Analisis III}
\author{Javier Vera}
\maketitle
\newpage

\section{Cayley Hamilton Generalizado}
\begin{theorem}[Cayley Hamilton Generalizado]

  Sea $\mathbb{V}$ un espacio vectorial de dimensión finita y $T: \mathbb{V} \rightarrow \mathbb{V}$
  \begin{enumerate}
    \item $m_T | p_T$ además tienen los mismos factores primos
    \item Si $m_T = p_1^{v_1}\ldots p_n^{v_n}\quad \land\quad p_T = p_1^{d_1}\ldots p_n^{d_n}$ entonces:
      $$d_i = \frac{dim(Nu(p_i)^{v_i})}{gr(p_i)}$$
  \end{enumerate}
  \begin{proof}
    \begin{enumerate}
      \item Primero usamos descomposición cíclica, obtenemos $p_1,\ldots p_n$ anuladores de $v_1,\ldots v_n$ con, $p_T = p_1\ldots p_n$ tales que $p_i | p_{i-1}$ y  
        $$ \mathbb{V} = \bigoplus_{i = 1}^{n} Z(v_i , T_i) $$ 

        Además sabemos que $m_T = p_1$ entonces $m_T = p_1 | p_1\ldots p_n = p_T$ 
        Ahora queremos ver que $p$ primo divide a $p_T$ si y solo si divide a $m_T$

        Sea $p|m_T$ entonces obviamente divide a $p_T$

        Sea $p|p_T = p_1\ldots p_n$ entonces existe $i$ tal que $p|p_i$ entonces $p|p_i \land p_i|p_1=m_T$
      \item Calculemos $r_i$ usando descomposición primaria tenemos $$\mathbb{V} = \bigoplus_{i=1}^{n} V_i $$ 

        con $V_i = Nu (p_i(T)^{v_i})$ T-invariantes y además $T_i := T|_{V_i} : V_i \rightarrow V_i$ con $m_{T_i} = p_i^{v_i}$ 
        y por parte 1 tenemos $p_{T_i} = p_i^{r_i}$ con 
        $r_i \geq v_i$ además también por T-invarianza de los $V_i$, $p_T = p_{T_1}\ldots p_{T_n}$ entonces 
        $$p_1^{d_1}\ldots p_n^{d_n} = p_T = p_{T_1}\ldots p_{T_i} = p_1^{r_1}\ldots p_n^{r_n}$$

        Por lo tanto $d_i = r_i$. Pero entonces $ dim(Nu(p_i(T)^{v_i})= dim(V_i) = gr(p_{T_i}) = gr(p_i^{d_i}) = gr(p_i).d_i$
    \end{enumerate}
  \end{proof}
\end{theorem}

\section{Diagonal + Nilpotente}
\begin{theorem}[Diagonal + Nilpotente]
  Sea $T: \mathbb{V}\rightarrow \mathhbb{V}$ con $dim\mathbb{V} \leq \infty$ tal que $m_T$ es producto de factores lineales entonces
  \begin{enumerate}
    \item $T = D+N$
    \item $DN = ND$
  \end{enumerate}

  \begin{proof}
    \begin{enumerate}
      \item Como $m_t=(x-c_1)^{d_1}\ldots (x-c_n)^{d_n} $ es producto de factores lineales es producto de factores primos usamos teorema descomposición primaria y tenemos
        $V = \bigoplus V_i$ con $V_i = Nu((T-c_i)^{d_i})$. Además tenemos $E_i$ que proyectan en cada $V_i$ y por descomposición primaria
        estos son polinomios evaluados en $T$.

        Sea $$D= c_1E_1 + \cdots c_nE_n \text{ obviamente diagonal }$$ 
        $$T.I = T(E_1 + \cdots + E_n) = TE_1 + \ldots TE_n$$

        Luego llamamos $N = T - D = (T-c_1)E_1 + \ldots (T-c_n)E_n$ notar que $N$ un pol evaluado en $T$ (por que $E_i$ lo son)

        Entonces conmuta con $D$ que también es un pol evaluado en $T$ entonces tenemos la parte ii. Ya tenemos el operador diagonal $D$
        nos falta ver que $N$ es nilpotente.

        $$ N^2 = (T-D)^2 = (\sum_{i=1}^n (T-c_i)E_i)^2 = \sum_{i,j=1}^n (T-c_i)E_i.(T-c_j)E_j =\sum_{i,j=1}^n (T-c_i)(T-c_j)E_jE_i = 
        \sum_{i=1}^n (T-c_i)^2E_i$$

        Usando propiedades de proyector. Ahora inductivamente es directo ver que $$N^r = \sum_{i=1}^n (T-c_i)^rE_i$$

        Ahora si tomamos $r = \max \{d_i: i=1\ldots n\}$ tenemos que $$N^r(v) = \sum_{i=1}^n (T-c_i)^rE_i(v)$$

        Pero $E_i(v) \in V_i = Nu((T-c_i)^{d_i})$ entonces $N^r(v) = 0 \quad \forall v \in \mathbb{V}$ mostrando que $N$ es nilpotente 

        Ahora resta ver unicidad. Sean $N', D'$ que cumplen las hipótesis entonces 
        $$D'T = D'(D'+N') = D'D' + D'N' = D'D' + N'D' = (D'+N')D'=TD' $$ por lo tanto $D'$ conmuta con $T$, análogamente
        $N'$ conmuta con $T$ tambien entonces obviamente conmutan con $D,N$ que son polinomios evaluados en $T$
        Sabemos que $N'+D' = T  = N + D$ entonces $N' - N = D' - D$. Por un lado $D,D'$ son
        ambas diagonalizables y como conmutan son simultaneamente diagonalizables entonces $D' - D $ es diagonalizable
        Sean $r,s$ tales que $N'^s = N^r = 0$
        $$(N' - N)^{r+s} = \sum_{i=0}^{r+s} \binom{r+s}{i} (-1)^{i}N^{r+s - i}N'^i = 
        \sum_{i=0}^s \binom{r+s}{i} (-1)^{i}N^{r+s - i}N'^i + \sum_{i = s+1}^r+s \binom{r+s}{i} (-1)^{i}N^{r+s - i}N'^i$$

        El primer sumando es $0$ por que $r+s-i \geq r \quad \forall i\leq s$ y $N^r = 0$ y el segundo es cero por que $i > s$
        y $N'^s = 0$ 

        Entonces $N'-N$ es nilpotente entonces su único autovalor es $0$, pero también es diagonal por que
         $N'-N = D' - D$ entonces tiene que ser el operador 0

        Entonces $N=N' \land D=D'$
    \end{enumerate}
  \end{proof}
\end{theorem}

\section{Caracterización de operadores Diagonalizables}
\begin{theorem}[Caracterización Diagonalizable]
  $T: V \rightarrow V$ con $dim(V) \leq \infty$ tal que $m_T = (x-c_1)\ldots (x-c_n)$ es producto de factores lineales si y solo si
  $T$ es diagonalizable.

  \begin{proof}
    (Ida) Sea $p = (x-c_1)\ldots(x-c_n)$, con $c_1,\ldots,c_n$ los autovalores de $T$ Sabemos que $(x-c_i)$ es raiz del carácteristico por lo tanto es raíz del minimal, por que tienen
    las mismas raices entonces $(x-c_i)|m_T \quad \forall i=1\ldots n$ entonces $p|m_T$


    Además sabemos que existe base de autovectores $\{v_1,\ldots,v_j\}$ para cada auto vector existe un aval $c_i$ tal que 
    $Tv_i = c_{j(i)}v_i$ equivalentemente $Tv_i - c_{j(i)}v_i = 0$ equivalentemente $p_i(T)(v_i) = 0$ con $p_i = (x-c_{j(i)})$

    Ahora tomemos cualquier $v_i$ en la base de autovectores $p_i(T)(v_i) = 0$ y es claro que existe $h\in K[x]$ tal que $p = p_i.h$ 

    Por lo tanto $p(T)(v_i) = 0$ entonces $m_{T,v_i}|p$ y esto vale para todo $i=1,\ldots,j$ entonces 
    $$m_T = mcm\{m_{T,v_1},\ldots,m_{T,v_j}\} |p $$ Mostrando finalmente $$p=m_T$$

    (Vuelta) Tomemos $W = W_1 \bigoplus \cdots \bigoplus W_n$ con $W_i$ auto espacio asociado a $c_i$. Supongamos $W\neq V$ entonces 
    tomo $v\in V- W$. Por lema anteriór sabemos que $\exists c_i$ autovalor tal que $w = (T-c_i)v \in W$. Entonces podemos escribir
    $(T-c_i)v = w = w_1 + \cdots + w_n$ con $w_i \in W_i$ respectivamente. 

    Ahora definimos $g = \frac{m_T}{x-c_i}$ y definimos $g - g(c_i) \in K[x]$ que claramente tiene
    como raíz a $c_i$ entonces lo podemos escribir como $h(x-c_i)$. Luego $$ g(T)v - g(c_i)v = h(T)(T-c_i)v = h(T)(w)$$

    Sabemos que $W$ es T-invariante entonces $h(T)(w) \in W$ por que un polinomio en $T$ no es mas que una combinacion de $T$

    Y $0 = m_T(T)(v) = (T-c_i)(g(T)(v))$ pero entonces $g(T)(v)$ es autovector de autovalor $c_i$ por lo tanto $g(T)(v) \ in W$

    Mostrando asi que $g(c_i)v \in W$ pero $v \notin W$ entonces $g(c_i) = 0$. Que es absurdo por como definimos a $g$ claramente $c_i$
    no puede ser raíz. El absurdo provino de suponer que existia un $v\in V - W$ entonces no existe dicho $v$ por lo tanto $V = W$ 
    mostrando que la suma de los autoespacios asociados son todo el espacio, por lo tanto tenemos una base de autovectores y $T$
    es diagonalizable
  \end{proof}
\end{theorem}

\section{Adjuntas}
\begin{theorem}[Proposición adjuntas]
  Dado $f \in \mathbb{V}^*$ tenemos que $\exists ! w \in \mathbb{V}$ tal que $$f(z) = (z|v) \quad \forall z \in \mathbb{V}$$
  \begin{proof}
    Sea $B = \{v_1,\ldots v_n\} $ base de $\mathbb{V}$, proponemos $v = \sum_{i=1}^n \overline{f(v_i)}v_i$. Probemos que funciona
    $$ (z|v) = (z|\sum \overline{f(v_i)}v_i) = f(v_i)(z|\sum v_i)  = f(\sum(z| v_i)v_i) = f(z) \quad \forall z\in\mathbb{V}$$

    Obs aca usamos que $\sum (z|v_i)v_i$ son coordenadas de $z$ en base $B$.

    Veamos que es único supongamos tenemos $v,v' \in \mathbb{V}$ tal que $$(z|v) = f(z) = (z|v')$$
    Entonces $$(z|v-v') = 0 \quad \forall z \in \mathbb{V}$$ en particular $$(v-v'|v-v')=0$$
    Mostrando que $$|| v-v'|| = 0 \iff v-v' = 0 \iff v = v'$$
  \end{proof}
  
\end{theorem}

\begin{theorem}[Transformaciones Adjuntas]
  Dada $T:V \rightarrow V$ existe una única transformacion lineal $T^*$ que cumple $$(Tv|w) = (v|T^*w)$$ La llamamos adjunta
  \begin{proof}
    Dada $T$ definimos $f_w(v) = (Tv|w)$ luego  $f_w\in \mathbb{V}^*$. Por proposición sabemos que 
    $$\exists ! w^* \text{ tal que } f_w(z) = (z|w^*) \quad \forall z \in \mathbb{V}$$

    Entonces $(Tv|w) = f_w(v) = (v|w^*)$. por lo tanto para cada $w\in \mathbb{V}$ tenemos un único $w^* \in \mathbb{V}$. Luego,
    naturalmente podemos definir 
    $$T^* :\mathbb{V} \rightarrow \mathbb{V} \quad T^*(w) = w^*$$
    Mostrando finalmente que $$(Tv|w) = (v|T^*w) \quad \forall v,w \in \mathbb{V}$$
  \end{proof}
\end{theorem}

\section{Descomposición Primaria}
\begin{theorem}[Descomposición Primaria]
  Sea $T: \mathbb{V} \rightarrow \mathbb{V}$ y $m_T=p_1^{r_1}.\ldots p_n^{r_n}$ producto de factores primos, entonces
  \begin{enumerate}
  \item $\mathbb{V} = V_1 \oplus \cdots \oplus V_n \text{ con } V_i = Nu(p_i(T)^{r_i})$
  \item Además si $T_i = T|_{V_i} : \mathbb{V}_i \rightarrow \mathbb{V}_i $ sucede que $m_{T_i} = p_i^{r_i}$. 
  Con $V_i$ T-invariante
  \end{enumerate}
  
  \begin{proof}
      Sea $$f_i = \frac{m_T }{p_i^{r_i}} = \prod_{i\neq j } p_j^{r_j}$$

      Claramente $f_i$ son coprimos entonces $\exists g_i \in \mathbb{K}[x]\ / \ 1 = f_1g_1 + \cdots f_ng_n$
      (Por que como son coprimos el generador de su ideal es el 1). Entonces definimos $E_i = f_ig_i(T)$.

      Primero notemos que $$E_1 + \cdots + E_n = f_1g_1(T) + \cdots + f_ng_n(T)= (f_1g_1 + \cdots + f_ng_n)(T) = 1(T) = I$$. 

      Ahora $E_iE_j = f_ig_if_jg_j(T) = g_ig_jf_if_k(T)= 0 $ por que $f_if_j$ es obviamente un múltiplo del minimal de T si $j\neq i $
      (por como definimos $f_i$). Entonces $E_iE_j = 0$

      
      Por lo tanto sabemos que es directo concluir de aca obviamente sale $E_i^2 = E_i$

      Luego como cumple estas propiedades $\mathbb{V }  = V_1' \oplus \cdots \oplus V_n' \text{ con } V_i' = Im(E_i)$

      Y por otro lado $E_i$ son pols evaluados en T por lo tanto conmutan con T luego 
      $$E_i T = TE_i \quad \forall i=1,\ldots,n$$ Mostrando que los $V_i$ son T-invariante
      
      Ahora queremos ver que $Im(E_i) = V_i' = Nu(p_i^{r_i}(T)) = V_i$. Tomemos $v \in V_i'$ luego $E_i(v) = v$ por lo tanto 
      $$ p_i^{r_i}(T)(v) = p_i^{r_i}(T)E_i(v) = p_i^{r_i}(T)f_ig_i(T)(v)= g_im_T(T)(v) = g_i m_T(T)(v) = 0.v = 0$$
      Mostrando que $v \in V_i$. Ahora tomemos $v \in V_i$. Sabemos que $p_i^{r_i} | f_jg_j \quad \forall j \neq i$ entonces 
      $f_jg_j = p_i^{r_i}.h$ por lo tanto $$E_j(v) = p_i^{r_i}h(T)(v) = 0  $$ 
      por que $v\in Nu(p_i^{r_i}(T))$ y esto vale $\forall j \neq i$ entonces 
      $$v = v.I = v(E_1 + \cdots + E_n) = E_1(v) + \cdots + E_n(v) = E_i(v)$$

      Mostrando que $v \in Im(E_i) = V_i'$ Completando asi la parte i

      Ahora notemos que $p_i^{r_i}(T)(v) = 0 \quad \forall v \in V_i $ por lo tanto $m_{T_i} | p_i^{r_i}$ y 
      $f_i(T)v = 0 \quad \forall v \in V_j \ j\neq i $. 

      Ahora sea $g$ tal que $g(T_i)(v) = 0 $ entonces $g(T)f_i(T)(v) = 0\quad \forall v \in \mathbb{V} $

      Mostrando que $p_i^{r_i}f_i = m_T | gf_i$ entonces como estamos hablando de factores primos $p_i^{r_i}|g$ como vale
      para cualquie polinomio anulador vale para $m_{T_i}$, entonces $m_{T_i}|p_i^{r_i}$, finalemente $m_{T_i} = p_i^{r_i}$
          
  \end{proof}
\end{theorem}

\begin{theorem}[Caracterización de Triangulables]
  Sea $ T: \mathbb{V} \rightarrow \mathbb{V}$ entonces su minimal se escribe en factores 
  primos si y solo si es triangularizable.
  \begin{proof}
      Ida. Como es triangularizable sabemos que en alguna base es una matriz triangular por lo tanto su característico
      se factoriza en factores primos, por Cayley-Hamilton, el minimal se factoriza en factores primos

      Vuelta. Sea $m_T = p_1^{r_1}\ldots p_i^{r_i}$ para empezar seguro tenemos por lo menos un autovalor $c_1$,
      entonces definimos $V_1 = <v_1>$ con $v_1$ auto vector de autovalor $c_1$. Por lo tanto $V_1$ es T-invariante.
      Ahora como el minimal se factoriza en factores primos y $V_1$ T-invariante, entonces podemos aplicar lema.
      $$\exists v_2 \notin V_1 \text{ tal que } (T-c_2I)v_2 = v \in V_1$$.Llamamos $V_2 = <v_1,v_2>$. Obviamente $v_1$ y $v_2$ son
      lineamente independiente por que $v_2 \notin V_1$.

      Además $T(v_2) - cv_2 = v$ con $v\in V_1$ entonces $T(v_2) = cv_2 + v$ por lo tanto $T(v_2) \in V_2$

      Finalmente $V_2$ es T-invariante. Ahora hacemos esto recursivamente y llegamos a 
      $$ \exists v_k \notin V_{k-1} \text{ tal que } (T-c_k)v_k \in V_{k-1}$$

      Entonces tenemos $V_{k} = \{v_1,\ldots,v_k\}$ que es T-invariante y de $dim(V_k) = k$

      Si tomamos $k=n$ tenemos una base que claramente triangulariza por que $T(v_i)\in V_{i}$
  \end{proof}
\end{theorem}

\end{document}
