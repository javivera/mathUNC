\documentclass{article}

\usepackage{amssymb}
\usepackage{amsmath}
\usepackage{amsthm}
\usepackage{mathpazo}
\usepackage{tcolorbox}
\usepackage[margin=0.8in]{geometry}
\usepackage[colorlinks=true]{hyperref}
\usepackage{enumerate}   

\newtheoremstyle{break}
  {\topsep}{\topsep}%
  {\itshape}{}%
  {\bfseries}{}%
  {\newline}{}%
\theoremstyle{break}
\newtheorem{theorem}{Teorema}[section]
\newtheorem{corollary}{Corolario}[theorem]
\newtheorem{lemma}[theorem]{Lema}
\newtheorem{proposition}{Proposición}
\newtheorem*{remark}{Observación}
\newtheorem{definition}{Definición}[section]

\def \R{\mathbb{R}}
\def \N{\mathbb{N}}
\def \Z{\mathbb{Z}}

\begin{document}
    % LaTeX
    
    \title{Resumen 2do Parcial}
    \author{Javier Vera}
    \maketitle

    \section{Integrales}
    \begin{itemize}
		 \item Fracciones Simples (Si el pol de arriba mayor grado que el de abajo dividir)
		\item Atento a dividir polinomios por mas que quede resto
		\item Llevar polinomio de arriba a derivada de polinomio de abajo
				\[\int \frac{5x+2}{x^2+2x+10}dx = \int \frac{\frac{5}{2}(\frac{2}{5}5x+\frac{2}{5}2)}{x^2+2x+10}dx = \int \frac{\frac{5}{2}(2x+2-\frac{6}{5})}{x^2+2x+10}dx = \frac{5}{2}\int \frac{2x+2}{x^2+2x+10}dx -\int \frac{3}{x^2+2x+10}dx\]

				Y ahora podemos usar $u=x^2+2x+10 \quad du=(2x+2)dx$. Segundo sumando lo llevamos a cuadrado


		\item Sustitución. Si $x = 2\arctan(t)$ y usando $ t=\tan( \frac{x}{2} )$ obtenemos: 	
				$\cos (x) = \frac{1-t^2}{1+t^2} \quad \sin (x) = \frac{2t}{1+t^2} \quad dx = \frac{2}{1+t^2}dt$
				\[\text{Ejemplo}\quad 2x=\arctan (t) \quad dx=\frac{2}{1+t^2} \Rightarrow \int \frac{dx}{1+\cos(x)} = \int \frac{\frac{^2}{1+t^2}}{1+\frac{1-t^2}{1+t^2}}\]

		\item Completar cuadrados en el polinomio de abajo para llegar a algo del estilo $\int \frac{1}{u^2+1}$
				\[\int \frac{dx}{x^2+x+1} = \int \frac{dx}{(x+ \frac{1}{2})^2 + \frac{3}{4}} = \int \frac{du}{u^2 + \frac{3}{4}} 
				= \frac{4}{3}\int {\frac{du}{\frac{4}{3}u^2 + 1}} = \frac{4}{3} \int{\frac{du}{(\frac{2u}{\sqrt{3}})^2+1}} 
				= \int{ \frac{dv}{v^2 + 1}} = \arctan(v)\]
	
				Obs: A veces es mas facil y solo con un reemplazo llegamos a $\frac{1}{x^2+1}$

				\[\int \frac{dx}{x^2-6x+10} = \int \frac{dx}{(x-3)^2+1} = \int \frac{du}{u^2+1}\]

		\item Llevar a la forma $\int \frac{1}{\sqrt{1-x^2}}$
				\[\int \frac{x}{\sqrt{1-x^4}} = \frac{1}{2}\int \frac{du}{\sqrt{1-u^2}} = \arcsin(x)\]
		\item Partes

		\item Multiplicar y dividir
				\[\int \sec x = \int \frac{\sec (x) (\sec x + \tan x)}{\sec x+\tan x} = \int\frac{\sec^2 x+ \sec\tan x}{\sec x + \tan x }\]
				Y ahora usamos 
				\[u= \sec x + \tan x \quad du=\sec x \tan x + \sec^2x\]
				A veces multiplicar por `conjugado`

		\item Salvar raices:
				\[\int \frac{1}{x^{\frac{1}{2}}+ x^{\frac{1}{3}}} = 6\int \frac{u^5}{u^3+u^2} \quad \text{ usando }\quad x = u^6 \quad dx = 6u^5du\]
		
		\item Formulas de reduccion:
				\[\int \cos^n(x) dx = \frac{1}{n}\cos^{n-1}(x)\sin(x) + \frac{n-1}{n}\int\cos^{n-2}(x)dx\] 
				(La del seno es igual pero donde dice seno poner coseno y donde dice coseno poner seno)
				\[\int\frac{1}{(x^2+1)^n}= \int\frac{1}{2n-2}\frac{x}{(x^2+1)^{n-1}} +\frac{2n-3}{2n-2} \int\frac{dx}{(x^2+1)^{n-1}}\]
	
		\item Sustitución Trigonometrica
			\begin{enumerate}[(i)]
						\item Usando $\cosh^2 -\sinh^2 = 1$
								\[\int\frac{x}{\sqrt{1+x^2}}\quad x=\sinh t \quad dx=\cosh t \ \text{llegamos a } 
								\int \frac{\sinh t.\cosh t}{\sqrt{1+\sinh^2 t}} = \int\frac{\sinh t \cosh t}{\sqrt{\cosh^2 t}}\] 
						\item Usando $\cos^2+\sin^2=1$
								\[\int\frac{x}{\sqrt{1+x^2}}dx \text{ y usamos } x=\sin u \ dx=\cos u du \Longrightarrow 
						\int\frac{\cos u}{\sqrt{1-\sin^2u}} = \int\frac{\cos u}{\sqrt{cos^2u}} \]
				\end{enumerate}
		\item Integrales importantes:
				\[\int \frac{dx}{\sqrt{1-x^2}}= \arcsin (x) \quad \int \frac{1}{1+x^2}dx = \arctan x\]
				\[\sec x= \frac{1}{\cos x} \quad \quad \tan^2x = \sec^2(x) -1\]
				\[\int \sec x \tan x = \sec x \quad \int\sec^2(x)dx = \tan x\]
				La integral $\int \sec x$ ya la sabemos por ejercicios de practico
    \end{itemize}

	\section{Convergencia de Integrales}
	\subsection{Criterios}
	Sean $f,g$ dos funciones integrables en $[a,c] \quad \forall c\in [a,b]$ tales que $0 \leq f(x) \leq g(x) \quad \forall x\in [a,b]$
	\begin{itemize}
		\item  Comparación
			\[\int_{a}^{b} g(x)dx \text{ converge }\Rightarrow \int_{a}^{b} f(x)dx \text{ converge }\]
			Observación: También tenemos su análogo
		\item Cociente
			\[\lim_{x\rightarrow b^-} \frac{f(x)}{g(x)}= l \] 
			\begin{enumerate}[(i)]
				\item Si $0< l \leq \infty $ entonces:
					\[\int_{a}^{b} g(x)dx\ \text{ es convergente si y solo si}\ \int_{a}^{b}f(x)dx \text{ es convergente}\] 
				\item Si $l=0$ entonces: 
					\[\int_{a}^{b}g(x)dx\ \text{ es convergente entonces}\ \int_{a}^{b}f(x)dx\text{ es convergente}\]
				\item Si $l=\infty$ entonces:
					\[\int_{a}^{b}g(x)dx\ \text{ es divergente entonces }\ \int_{a}^{b}f(x)dx \text{ es divergente} \]
			\end{enumerate}

			Observación: Cambiando con cuidado el enunciado tenemos el mismo criterio con uno de los bordes a infinito
		\item Convergencia absoluta
			\[\int_{a}^{b}|f(x)|dx\ \text{ converge entonces}\ \int_{a}^{b}f(x)dx\ \text{ converge}\]
		\item Criterios P
			\[\int_{1}^{\infty}\frac{1}{x^p} \text{ converge a } \frac{1}{p-1}\ \text{ si }\ p>1\ \text{si no diverge}\ (p\leq1)\]	

			\[\int_{0}^{1}\frac{1}{x^p} \text{ converge a } \frac{1}{-p+1}\ \text{ si }\ p<1\ \text{si no diverge}\ (p\geq 1)\]	
	\end{itemize}
			
	\section{Series}
	\begin{itemize}
		\item Serie geométrica
			\[\sum_{n=0}^{\infty}ar^n = \frac{a}{1-r} \iff |r|<1\]
		\item $\{a_n\}$ es sumable $\Longrightarrow \lim_{n\rightarrow \infty}a_n = 0$
		\item Sea $a_n \geq 0\quad\forall n\in\N $ entonces $\sum_{n=1}^{\infty a_n}$ converge $\iff $ $\{S_n\}$ es acotada
		\item Si $a_n \geq 0 \quad\forall n\in\N$ entonces $\{S_n\}$ es creciente por lo tanto si es acotada converge
	\subsection{Criterios}
	\begin{itemize}
		\item Comparacion 
			\[\text{Sean } a_n,b_n \text{ sucesiones tales que } 0\leq a_n\leq b_n \quad \forall n\in\N\]
			\[\sum b_n \ \text{converge}\Longrightarrow \sum a_n \ \text{converge} \]
			\[\sum a_n \ \text{diverge}\Longrightarrow\sum b_n \ \text{diverge}\]
		\item Cociente. Sean $a_n,b_n$ sucesiones positivas $\forall n\in\N$ tales que $\lim_{n\rightarrow \infty}\frac{a_n}{b_n}= c$ 
			con $c\neq 0 \land c\neq \infty$ entonces:
			\[\sum a_n \ \text{converge}\iff \sum b_n \ \text{converge}\]
		\item Sea $a_n$ sucesion de terminos positivos tal que $\lim_{n\rightarrow \infty}\frac{a_{n+1}}{a_n}=r$ 
			\[r>1 \Longrightarrow\sum a_n\ \text{no converge}\]
			\[r<1 \Longrightarrow\sum a_n\ \text{converge}\]
		\item Integral. Sea $f$ positiva y decreciente en $[1,\infty]$ tal que $f(n)=a_n \quad \forall n\in\N$
			\[\int_{1}^{\infty}f(t)dt \ \text{converge}\iff \sum_{n=1}^{\infty}a_n \ \text{converge}\]
		\item Criterio Raiz. Si $\lim_{n\rightarrow \infty} \sqrt[n]{|a_n|} = l$
			\[l<1 \Longrightarrow \sum |a_n|\ \text{converge }\]
			\[l>1 \Longrightarrow\sum a_n \ \text{diverge}\]
			\[l=\infty \Longrightarrow\sum a_n \ \text{no converge}\]
		\item $\sum |a_n|$ converge entonces $\sum a_n$ converge. Ademas $\{a_{n+}\}$ y $\{a_{n-}\}$ convergen tambien
		\item Leibniz. Sea $a_n$ una sucesion decreciente y de terminos positivos tal que $\lim_{n\rightarrow \infty}a_n = 0$
			\[\sum_{n=1}^{\infty} (-1)^{n+1}a_n\ \text{converge}\]
		\item Raave. $a_n$ sucesion de terminos positivos tal que 
			\[r=\lim_{n\rightarrow \infty }n(1-\frac{a_n}{a_{n-1}})\]
			\[r>1 \Longrightarrow\sum a_n\ \text{converge }\]
			\[r<1 \Longrightarrow\sum a_n\ \text{diverge}\]
			Puede ser util cuando en el criterio de la raiz o en el del cociente nos da 1 el limite
	\end{itemize}
	\end{itemize}
	
	\section{Polinomios de Taylor}
	\begin{itemize}
		\item El polinomio de la suma de funciones es igual a la suma de polinomios de dichas funciones
		\item $(P_{n,a,f}(x))'=P_{n-1,a,f'}(x)$
		\item $P_{n,a,f.g}(x)=P_{n,a,f}.P_{n,a,g}(x)$
		\item $P_{n,a,cf}(x)=cP_{n,a,f}(x)$
		\item $P_{n,a,f+g}(x)=P_{n,a,f}(x)+P_{n,a,g}(x)$
		\item Sea $P_{n,a,f}=\sum_{i=0}^{n}a_i(x-a)^i$ entonces $a_n=\frac{f^{(n)}(x)}{n!}$ por lo tanto $n!a_n=f^{(n)}(x)$
		\item Sea $f(x)=g(x^n)$, $p(x)=P_{m,0,g}(x)$ y $q(x)=p(x^n)$ entonces $q(x)=P_{n.m,0,f}$ 
		\item Sea $f$ una función tal que $f,f^1, \cdots f^{n+1}$ estan bien definidas en $[a,b] $ y sea $R_{n,a}$ el resto del polinomio de taylor
			de grado n centrado en a:
			\begin{enumerate}[(i)]
				\item $R_{n,a}(x) = \frac{f^{n+1}(t)}{n!}(x-t)^n(x-a) \quad \text{ para algún }\ t\in(a,x)$
				\item $R_{n,a}(x) = \frac{f^{n+1}(t)}{(n+1)!}(x-a)^{n+1} \quad \text{ para algún }\ t\in(a,x)$
				\item $R_{n,a}(x) =\int_{a}^{x}\frac{f^{n+1}(t)}{n!}(x-t)^n dt\quad \text{ para algún }\ t\in(a,x)$
			    
			\end{enumerate}
		\item Dos funciones se dicen iguales en orden n al rededor de a si cumplen
			\[\lim_{x\rightarrow a}\frac{f(x)-g(x)}{(x-a)^n}=0\]
		\item Si $f$ es derivable n veces y $P$ es un polinomio que es igual a f hasta orden n al rededor de a, entonces es el polinomio de taylor de f
			de grado n
		\item Dada $f$
			\[ \lim_{x\rightarrow a} \frac{f(x)-P_{n,a,f}(x)}{(x-a)^n} = 0\]
	\end{itemize}

\end{document} 
