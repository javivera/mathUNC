\documentclass{article}

\usepackage{amssymb}
\usepackage{amsmath}
\usepackage{amsthm}
\usepackage{mathpazo}
\usepackage{tcolorbox}
\usepackage[margin=0.8in]{geometry}
\usepackage[colorlinks=true]{hyperref}

\newtheoremstyle{break}
  {\topsep}{\topsep}%
  {\itshape}{}%
  {\bfseries}{}%
  {\newline}{}%
\theoremstyle{break}
\newtheorem{theorem}{Teorema}[section]
\newtheorem{corollary}{Corolario}[theorem]
\newtheorem{lemma}[theorem]{Lema}
\newtheorem{proposition}{Proposición}
\newtheorem*{remark}{Observación}
\newtheorem{definition}{Definición}[section]

\begin{document}
    % LaTeX
    
\title{Title}
\author{Javier Vera}
\maketitle

\section{Ejercicio 4}
\begin{itemize}
	\item Falso.Por ejemplo definamos la siguiente transformacion
		\[
			T(e_1) = e_1 \quad T(e_2) = e_2 \quad T(e_3) = 2e_3
		\]
		Esta transformación lineal tiene autovalores $1$ y $2$ que son dos autovalores distintos. Y es directo ver que en base
		canonica esta transformación ya esta diagnolizada, por ende es diagonalizable
	\item Falso. Usando cualquier función $f:\mathbb{Q} \rightarrow \mathbb{Q}$ que separe con la suma pero no con escalar
	Por ejemplo 
\end{itemize}

\section{Ejercicio 5}
\begin{itemize}
	\item Calculando el determinante nos da $\det(A) = 4a-8b+12c$. Ahora probamos que es subespacio. Veamos que es cerrado para 
		suma. Tomando $\mathcal{T} = \{ (a,b,c)\in \mathbb{R}|\det(A)=0 \}  $   

		\begin{multline} 
			\text{Sean } (a_1,b_1,c_1)\land(a_2,b_2,c_2)\in \mathcal{T} \Rightarrow \\
		-4(a_1,b_1,c_1)-8(a_1,b_1,c_1)+12(a_1,b_1,c_1)=0 \land -4(a_2,b_2,c_2)-8(a_2,b_2,c_2)+12(a_2,b_2,c_2)=0\\
		\text{Ahora tenemos que }
		-4((a_1,b_1,c_1)+(a_2,b_2,c_2))-8((a_1,b_1,c_1)+(a_2,b_2,c_2))+12((a_1,b_1,c_1)+(a_2,b_2,c_2))\\
		= -4((a_1,b_1,c_1)-4(a_2,b_2,c_2))-8((a_1,b_1,c_1)-8(a_2,b_2,c_2))+12((a_1,b_1,c_1)+12(a_2,b_2,c_2))\\
		= [-4(a_1,b_1,c_1)-8(a_1,b_1,c_1)+12(a_1,b_1,c_1)] + [-4(a_2,b_2,c_2)-8(a_2,b_2,c_2)+12(a_2,b_2,c_2)]=0
	\end{multline}
	Entonces es cerrado para la suma

\end{itemize}
\section{Ejercicio 6}
	Si hacemos la matriz de dicha TL usando las base canonica $B = \{ 1,x\ldots x^{n} \} $ nos queda 
	\[
	[A]_{B} =
	\left[ {\begin{array}{ccccc}
	1 & -1 & 0 & \cdots & 0\\
	0 & 1 & -2 & \cdots & 0  \\
	\vdots & & \ddots & \ddots & 0\\
	 & & & \ddots & n-1\\
	0 & & & & 1\\
	\end{array} } \right]
	\]
	Como se triangular su det es la diagonal multipicada que es 1, que es diferente que 0, por lo tanto es inversible.
	Además es directo cacular su pol característico 
	\[ \mathcal{X}_{A}= {(\lambda -1)}^{n}\] 
	Reemplazando en la matriz adecuada con el autovalor 1, y buscando su nucleo llegamos a 
	\[Nu(\lambda I - A) = <(1,0\ldots,0)>\] que en pols es $<1>$. Que tiene sentido, son los únicos pols que al aplicarles T vuelven ellos mismos. Por lo tanto no es diagonalizable, la dim del autoespacio no alcanza.

	\section{Ejercicio 7}
	(a) y (b) salen por propiedades de funciones y sumatorias

	\[
		\phi(T(f),T(g)) = \sum_{n=1}^{\infty} f_1(n)g_1(n) = 0 + \sum_{n=2}^{\infty} f(n-1)g(n-1) = \sum_{n=1}^{\infty} f(n)g(n) = \phi
		(f,g)
	\]
	(c) Cualquier sucesión que tenga algo diferente de 0 en el primer elemento y que sea constantemente 0 a partir de algun $n_0$

	\section{Ejercicio 8}
	(a) Ver que son li usando $B = \{\frac{x^2 -x}{2},-x^2+1,\frac{x^2 + x}{2}\}$

	(c) \[
		\phi = (\phi(\frac{x^2 -x}{2}),\phi(-x^2+1),\phi(\frac{x^2 + x}{2}))_{B^*}
	\]

\end{document}
