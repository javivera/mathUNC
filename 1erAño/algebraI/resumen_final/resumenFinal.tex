\documentclass{article}

\usepackage{amssymb}
\usepackage{amsmath}
\usepackage{amsthm}
\usepackage{mathpazo}
\usepackage{tcolorbox}
\usepackage[margin=0.8in]{geometry}
\usepackage[colorlinks=true]{hyperref}

\newtheoremstyle{break}
  {\topsep}{\topsep}%
  {\itshape}{}%
  {\bfseries}{}%
  {\newline}{}%
\theoremstyle{break}
\newtheorem{theorem}{Teorema}[section]
\newtheorem{corollary}{Corolario}[theorem]
\newtheorem{lemma}[theorem]{Lema}
\newtheorem{proposition}{Proposición}
\newtheorem*{remark}{Observación}
\newtheorem{definition}{Definición}[section]

\def \R{\mathbb{R}}
\def \N{\mathbb{N}}
\def \Z{\mathbb{Z}}

\begin{document}
    % LaTeX
    
    \title{Resumen Álgebra I}
    \author{Javier Vera}
    \maketitle

    \section{Relaciones}
    Dado un conjunto $A$ una relación es un suboconjunto $R$ de pares ordenados de $A$.
    Es decir $R \subseteq A \times A$. Dados $x,y \in A$ decimos que están relacionados si $(x,y) \in R$
    (Notación: $x \mathrel{R} y$ o $x \sim y$) \\
    Hay diferentes tipos de relaciones sobre un conjunto $A$:
    \begin{itemize}
        \item Reflexivas: $\forall x \in A \quad x\sim x$
        \item Simétricas: $\forall x,y \in A \quad x\sim y \Rightarrow y \sim x$
        \item Antisimétrica: $\forall x,y \in A \quad x\sim y \land y\sim x \Rightarrow x = y$
        \item Transitiva: $\forall x,y \in A \quad x\sim y \land y\sim z \Rightarrow x \sim z$
    \end{itemize}  
    \noindent Una relación Reflexiva, simétrica y transitiva se llama Relación de equivalencia

    \noindent Una relación Reflexiva, antisimétrica y transitiva se llama Relación de orden 

    \subsection{Particiones}

    Dada una relación de equivalencia $R$ en un conjunto $A$ definimos una partición 
    
    $$[a] = \{b \in A | b\sim a\} = \{b \in A | a \sim b \}$$

    Llamamos $a$ al representante de dicha partición. 

    \begin{lemma}
        Por un lado tenemos $a \in [a]$. Por otro lado dos clases de equivalencia $[a],[b]$ son disjuntas
        o son exactamente iguales
    \end{lemma}
    \begin{proof}
            Si son disjuntas no hay nada que demostrar, si no lo son entonces existe $c \in [a]\cap[b]$.
            Ahora tomemos cualquier $x \in [a]$ sabemos que $x\sim a$ pero entonces $x\sim c$ y luego $x\sim b$
            Entonces $x \in [b]$ por lo tanto $[a] \subseteq [b]$. \\
            Análogamente probamos $[b] \subseteq [a]$ Finalmente $[a] = [b]$
    \end{proof}
    
    \newpage
    
    \section{Número Naturales}
    \subsection{Axiómas de Peano}\label{sec:Peano}
    \begin{tcolorbox}
    \begin{itemize}
        \item[N1.] El 1 no és sucesor de nadie
        \item[N2.] Dos naturales distinto tienen distintos sucesores
        \item[N3.] (Axioma de inducción) Si $K \subseteq \N$ con $1 \in K$ y además dado $x \in K$
        sucede que $S(x) \in K$ (sucesors de x está en K) entonces $\N \subseteq K$
    \end{itemize}
    \end{tcolorbox}
    Hay otros axioma de Peano, pero no se han dado en la materia, se pueden buscar en Wikipedia

    \subsection{Inducción Matemática}\label{sec:Induc}
    Partiendo de los axiomas de Peano podemos definir $\N$ como el subconjunto de $\R$
    que cumple dichos axiomas
    \begin{tcolorbox}
        \begin{theorem}[Principio de inducción]
            Sea $P(n)$ una función proposicional con $n \in \N$
            \begin{enumerate}
                \item $P(1)$ es verdadera 
                \item Dado $n\in\N$ si $P(n)$ verdadera entonces $P(n+1)$ es veradera
            \end{enumerate}
        Entonces $P(n)$ es verdadera $\forall n\in \N$
        \end{theorem}
    \end{tcolorbox}
    \begin{proof}

            Sea $$K = \{n \in \N \ | \ P(n) \text{ es verdadera } \}$$ Por hipótesis 1. tenemos $1\in K$, por 2.
            tenemos que $n \in K$ implica $n+1 \in K$ usando estos dos se cumple el axioma N3 de peano. Por lo tanto
            $\N \subseteq K$ esto implica además sabemos que $K\subseteq \N$ entonces tenemos $K=\N$.
            Esto implica que $P(n)$ es verdadera $\forall n \in \N$
    \end{proof}
    
    \begin{tcolorbox}\label{sec:corrida}
        \begin{theorem}[Inducción Corrida]
            Sea $P(n)$ una función proposicional con $n \geq N$
            \begin{enumerate}
                \item $P(N)$ es verdadera
                \item Si $P(n)$ verdadera implica $P(n+1)$ verdadera $\forall n \geq N$
            \end{enumerate}

            Entonces $P(n)$ es verdadera $\forall n \geq N$
        \end{theorem}    
    \end{tcolorbox}

    \begin{proof}
        Sea $Q(n) = P(N -1 + k)$. Ahora consideremos el conjunto $$ K = \{n \in\N \ |\ Q(n) \text{ es verdadera }\}$$
        
        \noindent Trivialmente tenemos $Q(1) = P(N)$ entonces por hipótesis es verdadera. Además dado cualquier $n\in\N$ tenemos $Q(n) = P(N-1+n)$ y dado que $N-1 + n \geq N$
        por hipótesis $P(N-1+n+1) = Q(n+1)$ es verdadera.

        Pero entonces $K$ cumple las hipótesis de \hyperref[sec:Peano]{el axioma N3 de peano}, por lo tanto $\N\subseteq K$ y como sabemos $K \subseteq\N$
        entonces $K = \N$. 
        
        Como cada natural $m \geq N$ se escribe de la forma $N - 1 + n$ tenemos que $P(n)$ es verdadera $\forall n \geq N$
    \end{proof}

    \begin{tcolorbox}\label{sec:fuerte}
        \begin{theorem}[Inducción Fuerte]
            Sea $n\in\N$ y $P(n)$ una función proposicional
            \begin{enumerate}
                \item $P(1)$ es verdadera
                \item Si $P(1),P(2)\ldots P(k)$ verdaderas implica $P(k+1)$ verdadera $\forall k \in \N$
            \end{enumerate}

            Entonces $P(n)$ es verdadera $\forall n \in \N$
        \end{theorem}    
    \end{tcolorbox}


    \begin{remark}[Suma de Gauss]
        $$ \sum_{i=1}^{n} i = \frac{n.(n+1)}{2}$$

        \noindent Se demuestra usando inducción
    \end{remark}
    \begin{remark}[Suma Aritmética]
        Una sucesión es aritmética si satisface $a_{n+1} = a_n + c$. Por ejemplo $a_1 = b, \ a_2 = b + c, \ a_3 = b +2c \ \cdots a_n = b +c(n-1) $
        $$ \sum_{i=1}^{n} a_i = nb + \frac{(n-1)n}{2}$$
    \end{remark}    
    \begin{proof}
        $\sum_{i=1}^{n} a_i = \sum_{i=1}^{n} b + (i-1) = \sum_{i=1}^{n} b + \sum_{i=1}^{n} (i-1) = bn + \sum_{i=1}^{n} (i-1) = nb + \frac{(n-1)n}{2}$. 
        
        Aquí usamos la Suma de Gauss
    \end{proof}

    \begin{remark}[Suma Geométrica]
        Una sucesión es geométrica si satisface $a_1 = b, \ a_2 = br , \ a_3 = br^2 \cdots a_n = br^{n-1}$
        $$ \sum_{i=0}^{n} br^i = b\sum_{i=0}^{n} r^i  = b \frac{1-r^{n-1}}{1-r}$$
    \end{remark}

    
    \begin{lemma}
        $$ \text{Inducción Fuerte}\iff \text{Inducción Corrida} \iff \text{Inducción}$$
    \end{lemma}
    \begin{proof}
        Veamos $\text{Inducción Fuerte}\iff \text{Inducción}$

        $\Rightarrow )$ Tomemos $Q(k)$ como la proposición $"P(n) \text{ verdadera} \ \forall n \leq k"$. Para empezar $Q(1)$ es trivialmente verdadera.

        Ahora inducción nos dice que $Q(k)$ implica $Q(k+1)$ para cualquier $k \in\N$
        \begin{enumerate}
            \item Luego $" P(n) \text{ verdadera} \ \forall n \leq k " \iff Q(k)$ Verdadera.
            \item Pero si $Q(k)$ verdadera entonces $Q(k+1)$ Verdadera (Inducción)
            \item $Q(k+1) \text{ Verdadera } \iff "P(n) \text{ verdadera} \ \forall n \leq k+1"$ 
        \end{enumerate}

        Juntando nos queda $$P(1)\ldots P(k) \text{ Verdaderas} \iff "P(n) \text{ Verdadera} \ \forall n \leq k" \Rightarrow "P(n) \text{ Verdadera} \ \forall n \leq k+1" \Rightarrow P(k+1) \text{ Verdadera } $$ 

        Entonces por inducción corrida $P(n)$ es verdadera $\forall n \in \N$. Pero entonces 
        $Q(k)$ es verdadera $\forall k \in\N$.
        
        Por lo tanto probamos que si inducción completa vale.
        Las dos hipótesis de inducción implican el resultado de inducción que conocemos, que es lo mismo que decir que inducción vale.

        Probar que inducción completa vale usando inducción es trivial, queda como ejercicio para el lector 
        
    \end{proof}
    \newpage
    
    \section{Principio de buen órden}

    \begin{theorem}[Principio de buen órden]
        Un conjunto $A$ se dice bien ordenado si dado cualquier $B \subseteq A$ con $B\neq \emptyset$ tiene primer elemento.
    Un primer elmento es un elemento menor o igual que el resto. En esta matería solo nos interesará
    el buen órden de los naturales
    \end{theorem}
    \begin{proposition}
        $\N$ es bien ordenado
    \end{proposition}

    \begin{proof}
        Supongamos que no es bien ordenado entonces $\exists H \subseteq \N$ tal que $H \neq \emptyset$,
        sin primer elemento

        Ahora consideremos $H^c = \N - H $ sabemos que $1 \in H^c$, si nó estaría en $H$ lo cual sería
        absurdo

        Ahora consideremos $K = \{n \in \N \ | \ [1\ldots n] \subseteq H^c\}$. Por lo dicho arriba 
        $1\in K$

        Ademas si $n\in K$ tenemos que $n+1 \in K$ por que si $n+1 \notin K$ tendríamos que todos
        los elementos menores o iguales que $n$ estan en $K$ por lo tanto estarían en $H^c$ y 
        $n+1 \notin K$ por lo tanto $n+1 \notin H^c$ entonces $n+1 \in H$ pero entonces $n+1$ sería
        primer elemento de $H$ lo cual sería absurdo.

        Entonces $K$ cumple las hipótesis de inducción por lo tanto $K=\N$. Pero entonces $\N \subseteq H^c$
        Y ya teníamos que $H^c\subseteq \N$ por lo cual $H^c = \N$. Finamente $\N=\N - H$ entonces $H = \emptyset$. Absurdo

        Provino de suponer que existía dicho $H$ no vacío sin primer elemento, entonces no existe dicho
        conjunto
    \end{proof}
    \begin{proposition}
        Buen órden en $\N$ implica inducción
    \end{proposition}
    \begin{proof}
        Sea $K= \{n\in\N \ | \ P(n) \text{ es verdadera}\}$ y miremos usando las hipótesis de
        inducción y el buen orden nos gustaría llegar a que $K=\N$ o lo que es lo mismo 
        $K^c = \emptyset$. Supongamo que $K^c \neq \emptyset$ entonces por buen órden
        $\exists k \in K^c$ primer elemento. 

        Por inducción sabemos que $1 \in K$. Entonces dicho $k > 1$ luego tenemos que $n\in K$ para todo $n \leq k-1$.
        Si nó $k$ no sería primer elemento de $K^c$, pero por inducción esto nos dice que $k \in K$
        lo que es absurdo. Provino de suponer que $K^c \neq \emptyset$ por lo tanto $K^c=\emptyset$
        Entonces $K = \N$ probando que $P(n)$ es verdadera $\forall n\in\N$ y por lo tanto probando
        la implicación final de inducción
    \end{proof}

    \begin{proposition}
        Inducción implica buen órden en $\N$
    \end{proposition}
    \begin{proof}
        Sea $K \subseteq \N$ tal que $K\neq \emptyset$ y $K$ no tiene primer elemento.
        Como no tiene primer elemento $1 \in K^c$. Ahora supongamos que $1\ldots n \in K^c$ entonces
        $n+1 \in K^c$ por que si nó estaría en $K$ y sería su primer elemento, pero entonces por
        inducción $K^c = \N $ por lo tanto $K = \emptyset$ abusrdo. Provino de suponer que existía
        dicho $K\neq \emptyset$, no vacío y sin primer elemento, entonces todo $K\neq \emptyset$ 
        no vacío, tiene primer elemento 
    \end{proof}

    \begin{corollary}
        $$ \text {Buen órden en } \N \iff \text{ Inducción } \iff \text{Inducción completa} \iff \text{Inducción corrida}$$
    \end{corollary}
    
    \section{Combinatoria}

    \begin{definition}
        Un conjunto $A$ tiene $n$ elementos si y solo sí existe $$f: A \rightarrow [1\ldots n] \text{ biyectiva }$$
        En estos casos decimos $|A| = n$ (cardinal de $A$ es $n$)
    \end{definition}
    
    \begin{lemma}[Principio de adición]\label{sec:padicion}
        Sean $A$ y $B$ conjuntos disjuntos tales que $|A| = n$ y $|B| = m$ entonces $$|A \cup B| = |A| + |B|=n+m$$    
    \end{lemma}
    \begin{proof}
        Por hipótesis tenemos $f: A \rightarrow [1\ldots n]$ y $g: B \rightarrow [1\ldots m]$ ambas biyectivas. 

        Definimos $h: A\cup B \rightarrow [1\ldots n+m]$

        \[   
            h(x) = 
            \begin{cases}
            f(x) &\quad\text{si } x\in A \\
            g(x) +n &\quad\text{si } x\in B \\

            \end{cases}
        \]

        Veamos que es biyectiva. Podríamos ver que es inyectiva y suryectiva, pero es más facil dar su inversa.
        Sabemos que $f$ y $g$ tienen inversas por ser biyectivas llamemoslas $f^{-1}$ y $g^{-1}$ 

        Sea $r: [1\ldots n+m] \rightarrow A \cup B $
        \[   
            r(x) = 
            \begin{cases}
            f^{-1}(x) &\quad\text{si } x\leq n \\
            g^{-1}(x - n)  &\quad\text{si } x > n \\

            \end{cases}
        \]

        Veamos que efectivamente es su inversa mostrando que $h \circ r = id = r \circ h$

        Primero supongamos $x \leq n$ entonces $h\circ r (x) = h (r(x)) = h (f^{-1}(x))$ además $f^{-1}(x) \in A$ por definición 
        entonces $h(f^{-1}(x)) = f(f^{-1}(x)) = x$, por que $f^{-1}$ es la inversa de $f$.($f$ tiene
        inversa por ser biyectiva)
        
        Si en cambio $x > n$ entonces $h\circ r (x) = h (r(x)) = h (g^{-1}(x -n))$  además $g^{-1}(x) \in B$ por definición
        entonces $h(g^{-1}(x-n)) = g(g^{-1}(x-n)) + n = x-n+n = x$
            
        $r\circ h = id$ sale con las mismas ideas, queda como ejercicio para el lector.
        Entonces como $h(x)$ tiene inversa por lo tanto es biyectiva

        Finalmente $|A\cup B| = |[1\ldots n+m]| = n+m$
    \end{proof}

    \begin{corollary}
        $$|A_1 \cup A_2 \ldots \cup A_n| = |A_1| + |A_2| + \cdots + |A_n|$$
    \end{corollary}
    \begin{proof}
        Por inducción, el caso base es trivial.

        Ahora si tenemos $|A_1 \cup \ldots \cup A_n \cup A_{n+1}| = |(A_1\cup\ldots\cup A_n) \cup A_{n+1}| = |B \cup A_{n+1}| = |B| + |A_{n+1}|$

        Por hipótesis tenemos $|B| + |A_{n+1}| = |A_1| + |A_2| + \cdots + |A_n| + |A_{n+1}|$
    \end{proof}

    \begin{lemma}[Principio de multiplicación]\label{sec:pmult}
        $A$ y $B$ conjuntos finitos entonces $|A\times B| = |A||B|$
    \end{lemma}

    \begin{proof}
        Tenemos $A\times B = (\{a_1\} \times B )\cup  (\{a_2\} \times B )\cdots \cup(\{a_n\} \times B )$ 

        Por principio de adición 
        $|A\times B| = |\{a_1\} \times B |+ |\{a_2\} \times B | + \cdots +|\{a_n\} \times B |$

        Es trivial notar que $|\{a_n\} \times B|$ es igual para cualquier $n$

        Entonces calculemos $|\{a_1\} \times B|$. Sabemos que existe una función biyectiva $h: B\rightarrow [1\ldots|B|]$

        Entonces podemos construir la función $g: \{a_1\}\times B \rightarrow [1\ldots m]$ donde $|B| = m$
        dada por $$ g(a_1,b) = h(b)$$

        Veamos $g$ inyectiva. Sean $g(a_1,b_1) = g(a_1,b_2)$ entonces $h(b_1) = h(b_2)$ como $h$ es inyectiva
        $b_1=b_2$

        Veamos $g$ sobreyectiva. Sea $y\in[0\ldots |B|]$ dado que $h$ es sobreyectiva existe $b_1 \in B$ 
        tal que $h(b_1) = y$ pero entonces $g(a_1,b_1) = h(b_1) = y$. Finalmente dado cualquier $y \in [1\ldots |B|]$
        se puede dar una preimagen por $g$. 

        Entonces $g$ es sobreyectiva mostrando que $|\{a_n\} \cup B | = |B|$. Ahora si juntamos esto con lo que
        habíamos visto por principio de adición tenemos
        $$|A\times B| = |\{a_1\} \times B |+ |\{a_2\} \times B | + \cdots +|\{a_n\} \times B | = |B| + |B| \ldots |B|$$ 

        Donde la cantidad de sumandos es igual a la cantidad de elementos en $A$ que es igual a $|A|$

        Entonces $|A\times B|= |A||B|$


    \end{proof}

    \begin{lemma}[Principio de complemento]
        Sea $A \subseteq \mathcal{U}$ entonces $|A| = |\mathcal{U}| - |A^c|$
    \end{lemma}

    \begin{lemma}[Principio de inyección de Cantor]
        Sean $A$ y $B$ finitos entonces si $|A| > |B|$ no existe ninguna función $f:A \rightarrow B$ 
        inyectiva. Mas aún si tenemos $f: A\rightarrow B$ inyectiva entonces $|A| \leq |B|$
    \end{lemma}

    \begin{proof}
        Tomemos el conjunto $H = \{n\in\N | \exists m \in \N , m<n \text{ y } f:[1\ldots n] \rightarrow [1\ldots m] \text { inyectiva }\}$

        Nos gustaria ver que dicho conjunto es vacío. Supongamos que nó, entonces tiene primer elemento 
        llamémoslo $h$.Sabemos que $1 \notin H$ por que no existe ningún natural menor que 1. Entonces 
        $h > 1$

        Como $h \in H$ tenemos un $m \in\N$ con $m < h$ tal que $f:[1\ldots h] \rightarrow [1\ldots m]$
        inyectiva. Supongamos que $f(h) = m $, entonces podemos restringir $f$ 
        y obtener $g:[1\ldots h-1] \rightarrow [1\ldots m-1]$, que es inyectiva por que restringimos una inyectiva
        Pero entonces $h-1 \in H$ que sería absurdo.

        Ahora si $f(h) = p < m$ podemos armar armar $\alpha:[1\ldots m] \rightarrow [1\ldots m]$ con 
        $\alpha(p) = m$, $\alpha (m) = p$, $\alpha (i) = i \ \forall i\neq m \neq p$

        Ahora podemos dar $r = \alpha \circ f : [1\ldots h] \rightarrow [1\ldots m]$ 
        y sabemos que $r(h) = \alpha(f(h)) = \alpha(p) = m$. Pero entonces estamos en el anteriór caso
        restringimos $r$ y llegamos a un absurdo
    \end{proof}

    \begin{corollary}
        $|A| = n \land |A| = m$ entonces $n=m$
    \end{corollary}
    \begin{proof}
        $|A|=n$ entonces tenemos $f:A\rightarrow [1\ldots n]$ biyectiva y $g:A\rightarrow [1\ldots m]$

        Entonces tenemos $g \circ f^{-1} :[1\ldots n] \rightarrow [1\ldots m] $ inyectiva por ser composición
        de inyectivas entonces $$n = |[1\ldots n]| \leq |[1\ldots m]| = m$$ analogamente tenemos una función 
        $h:[1\ldots m] \rightarrow [1\ldots n]$ por lo cual $$m = |[1\ldots m]| \leq |[1\ldots n]| = n $$

        Entonces finalmente $n = m $
    \end{proof}

    \begin{remark}[Permutaciones]
        Sea $A$ tal que $|A| = n$ entonces hay $n!$ formas de ordenar $A$. En este tipo de conteo importa el órden
    \end{remark}

    \begin{remark}[Combinaciones]
	    Sea $A$ tal que $|A| = n$ entonces tenemos $\binom{n}{k}$ subconjuntos de $A$ te tamaño $k$. En este tipo de conteo
        no importa el ordan $\{1,2,3\}$ es lo mismo que $\{1,3,2\}$ por ende no lo cuento dos veces
    \end{remark}
    
    \begin{proof}
        Tengo que elejir $k elmentos$ para el primer elemento tengo $n$ opciones , para el segudno $n-1$ así hasta el 
        k-esimo en donde tengo $n-k+1$ opciones. Por ende tengo $n.(n-1)\ldots(n-k+1)$ formas de elegir $k$ elementos entre $n$ elementos
        Una vez seleccionados esos $k$ elementos, tengo $k!$ permutaciones dentro de mi selección que no me interesan dado que dos conjuntos
        con diferente orden son iguales. Entonces divido por $k!$ obteniendo 
        $$\frac{n.(n-1)\ldots(n-k+1)}{k!} = \frac{n.(n-1)\ldots(n-k+1)(n-k)!}{k!(n-k)!} = \frac{n!}{k!(n-k)}$$
    \end{proof}

    \begin{remark}[Arreglos]
        Un arreglo es una selección de $k$ elementos sobre un conjunto $A$. Si $|A|=n$ entonces tenemos $\frac{n!}{(n-k)!}$ arreglos
    \end{remark}

    \begin{remark}
	    $\binom{n}{k}$ = $ \binom{n}{n-k}$
    \end{remark}
    \begin{proof}
        $\binom{n}{n-k}$ = $\frac{n!}{(n-k)!(n-(n-k))!} = \frac{n!}{(n-k)!(k!)}$ = $\binom{n}{k}$        

        Otra forma alternativa es considerar $f: P_{k} (I_n) \rightarrow P_{n-k} (I_n)$ , la función complemento.
        Tomas conjuntos de tamaño $k$ y devuelve su complemento que son conjuntos de tamaño $n-k$. Resta ver que es biyectiva
        y eso nos diría que $\binom{n}{k}$   $=|P_{k}(I_n)| = |P_{n - k}(I_n)| = $ $\binom{n}{n-k}$  

    \end{proof}
    
    \begin{theorem}[Identidad de pascal]

	    $\binom{n+1}{k}$  = $\binom{n}{k-1}$   + $\binom{n}{k}$   
    \end{theorem}

    \begin{proof}
        Una forma es aritmética, queda como ejercicio para el lector

        Otra forma es mirar los conjuntos $$B = \{K \subseteq I_{n+1} \ | \ |K|=k \land n+1 \in K\}  = \{K \subseteq I_{n} \ | \ |K|=k-1\} = \mathcal{P}_{k-1}(I_n)$$
         $$A = \{K \subseteq I_{n+1} \ | |K| = k \land n+1 \notin K\}  = \{K \subseteq I_{n} \ | \ |K|=k \} = \mathcal{P}_{k}(I_n)$$

	 Notemos que $A\cup B = \mathcal{P}_{k}(I_{n+1})$ entonces $|A\cup B| =$ $\binom{n+1}{k}$  
        
	 Pero además tenemos $\binom{n+1}{k}$  $= |A\cup B| = |A| + |B| = |\mathcal{P}_{k-1}(I_n)| + |\mathcal{P}_{k}(I_n)|  = $ $\binom{n}{k-1}$   + $\binom{n}{k}$  
    \end{proof}

    \begin{theorem}
        $|\mathcal{P}(I_n)| = 2^n$. Mas aún dado $A$ tal que $|A|=n$ entonces $|\mathcal{P}(A)| = 2^n$
    \end{theorem}

    \begin{proof}
        Ahora notemos que $|\{f \ |\  f: I_n \rightarrow \{0,1\}\}| = 2^n$. Esto sale facil usando combinatoria, particularmente permutaciones
        dado el primer elemento de $I_n$ tenemos 2 opciones y así sucesivamente entonces tenemos $2.2.2\ldots 2$ n veces cantidad de funciones 
        en dicho conjunto

        Ahora veamos $g: \mathcal{P}(I_n) \rightarrow \{f \ |\  f: I_n \rightarrow \{0,1\} \}$ dada por
        \[   
            X \rightarrow f_X(x) \quad f_X(x) =
            \begin{cases}
            0 &\quad\text{si } x \notin X \\
            1  &\quad\text{si } x \in X \\

            \end{cases}
        \]

        Inyectividad dados $X, Y$ supongamos que $g(X) = f_x = f_y = g(Y)$.  Ahora tomemos $x \in X$ sabemos entonces que $f_x(x) = 1 $
        entonces como $f_x = f_y$ sabemos que $f_y(x) = 1$ por lo tanto $x\in Y$ esto nos dice $X \subseteq Y$ y analogamente vemos que 
        $Y \subseteq X$ mostrando que $X=Y$ mostrando inyectividad

        La sobreyectividad es trivial, dada cualquier función $f_x$ en $Im(g)$ podemos tomar $X$ y sabemos que $g(X) = f_x$ 

        Finalmente $g$ es biyectiva, por lo tanto $|\mathcal{P}(I_n)| = |\{f \ |\  f: I_n \rightarrow \{0,1\}\}| = 2^n$

        Hay otra prueba alternativa, veámosla   
        Sabemos que $|\mathcal{P}(I_n)| = \sum_{i=0}^{n} |\mathcal{P}_{i}(I_n)| = \sum_{i=0}^n $ $\binom{n}{i}$  

        Veamos que $\sum_{i=0}^n $ $\binom{n}{i}$   $= 2^n$ por inducción. El caso base es trivial

        Como hipotesis inductiva tenemos $\sum_{i=0}^n $ $\binom{n}{i}$   $= 2^n$. Miremos 

	\[ \left[\sum_{i=0}^{n+1} \binom{n+1}{i}\right]   = [\sum_{i=1}^n  \binom{n+1}{i}] + \binom{n+1}{n+1}   + \binom{n+1}{0} \]  

        Usando la identidad de pascal \[\left[\sum_{i=1}^n \binom{n+1}{i}  \right] + \binom{n+1}{n+1}   +\binom{n+1}{0}   = 
	\left(\sum_{i=1}^n  \binom{n}{i} + \binom{n}{i-1} \right)+ \binom{n+1}{n+1} + \binom{n+1}{0} \]

	Separando la suma \[ \left(\sum_{i=1}^n  \binom{n}{i} + \sum_{i=1}^n  \binom{n}{i-1} \right)+ \binom{n+1}{n+1} + \binom{n+1}{0} \]
	
	Ademas sabemos \[ \binom{n+1}{0}  =\binom{n}{0} \quad \land \quad \sum_{i=1}^n  \binom{n}{i-1}  = \sum_{i=0}^{n-1} \quad\land\quad  \binom{n}{n}  
	 \binom{n+1}{n+1} = \binom{n}{n} \]
        
	Juntando todo con cuidado queda 
	\[\sum_{i=0}^{n+1}  \binom{n+1}{i} = \sum_{i=0}^n  \binom{n}{n}   + \sum_{i=0}^n  \binom{n}{n} \] 
	
	Que por hipótesis es $2^n +2^n = 2.2^n = 2^{n+1}$. Probando así la inducción
    \end{proof}

    \begin{theorem}[Binomio de Newton]
 
    \[(x+y)^n = \sum_{k=0}^{n} \binom{n}{k}  x^{n-k}y^k \]
    \end{theorem}        

    \begin{proof}
        Sabemos que $(x+y)^n$ es una suma de productos de $x$ con $y$ con cada producto $x^{n-k}y^k$ con $0 \leq k \leq n$

	Ahora la pregunta es cuantos de cada uno de estos productos tenemos, primero seleccionamos $\binom{n}{k}$ esto nos dice
        las posiciones de las $y$ en el producto y una vez definidas estas lo espacios restantes son para $x$

	Entonces sabemos que tenemos $\binom{n}{k}$ formas de armar un producto de la pinta $x^{n-k}y^k$.

	Ahora si sumamos todo tenemos $\sum_{0}^{n}$ $\binom{n}{k}$ $x^{n-k}y^k$ como queríamos mostrar
    \end{proof}

    \section{Aritmética Entera}    

    Definimos a $\Z = -\N \cup \{0\} \cup \N$

    Y damos dos operaciones $(Producto) *: \Z\times \Z \rightarrow \Z$ y suma $(Suma) +: \Z\times \Z \rightarrow \Z$

    Estas operaciones las definimos como la restricción de las mismas en $\R$

    Tenemos entonces que $(\Z,(*,+))$ es un anillo conmutativo. Más aún es un anillo conmutativo al que le podemos dar un orden estricto
    \newpage
    \begin{definition}
        En todo anillo ordenado valen:
        \begin{enumerate}
        \item $a<b \Rightarrow a +c < b + c$
        \item $c>0 \land a >b \Rightarrow ca > cb$
        \item $-a.b = -(ab)$
        \end{enumerate}
    \end{definition}
     

    \begin{proposition}
        En todo anillo ordenado valen:
        \begin{enumerate}
            \item $c>0 \iff -c <0$
            \item $c<0 \land a >b \Rightarrow ac < bc$
            \item $ab=0 \iff a=0\lor b=0$
            \item $c\neq 0 \land ac = bc \Rightarrow a = b$
        \end{enumerate}
    \end{proposition}

    \begin{proof}
        \begin{enumerate}
            \item $c>0 \iff c -c  > 0 -c \iff 0 > -c$ (usando la prop 1 dada por órden)
            \item $c<0 \Rightarrow -c >0 \land a > b\Rightarrow -c.a > -c.b \iff -(ca) > -(cb) \iff cb > ca$ 
            

            (usando la propiedad 2 y 3 dadas por órden)
            \item Ida. Supongamos que $a\neq 0 \land b\neq 0$
            
            Hay que dividir en casos \begin{enumerate}
                \item $a>0 \land b>0 \Rightarrow ab > b.0 \iff ab>0$
                \item $a>0 \land b<0 \Rightarrow a.b < b.0 \iff ab < 0$
                \item Los otros dos caso son iguales
            
                Finalmente en todos los casos llegamos a que $ab \neq 0$ que es absurdo, provino de suponer $a\neq 0 \land b\neq 0$

                Entonces vale el complemento $a=0 \lor b=0$
            \end{enumerate}

            La vuelta es trivial
            \item $ac = bc \Rightarrow ac -bc = 0 \Rightarrow c(a-b) = 0$ por la prop anteriór 
            $c= 0 \lor a-b=0 \Rightarrow a-b = 0 \Rightarrow a=b$
        \end{enumerate}    
    \end{proof}

    \section{Divisibilidad}
    \begin{definition}
        Decimos que $a$ divide a $b$ sii $\exists k\in \Z$ tal que $a.k = b $ lo notamos $a|b$ y este $k$ es único dado que si $ak =b =ak'$
        entonces $ak -ak' =0 \iff a(k-k') = 0 $ como $a\neq 0$ entonces $k=k'$. Si $a=0$ entonces $b=0$ en ese caso $k$
        no es necesariamente único, pero es un caso que no nos interesa mucho
    \end{definition}

    \begin{proposition}
        $\forall a,b \in \Z$ valen:
        \begin{enumerate}
            \item $1|\pm a \land -1|\pm a$
            \item $a|\pm a \land -a|\pm a$
            \item $a|b \iff a|\pm b \iff -a | \pm b$
            \item $a|0$
            \item $0|a \iff a = 0 $
        \end{enumerate}
    \end{proposition}
    Las demostraciones son triviales y salen usando la definición

    \begin{proposition}
        \begin{enumerate}
            \item $a|b \land b|c \Rightarrow a|c$
            \item $a|b \land b|a \Rightarrow b = a \lor a = -b$
            \item $a|b \land a|c \Rightarrow a|b\pm c$
            \item $a|b\pm c \land a|b \Rightarrow a|c$
            \item $a|b \Rightarrow a|bc$
        \end{enumerate}
    \end{proposition}

    \begin{definition}
        El conjunto de divisores de $n$ se nota $div(n) = \{d\in\Z \ | \ d|n \}$. Y cumple ciertas propiedades
        \begin{enumerate}
            \item Además $\{1,-1,n,-n\} \subseteq div(n)$
            \item Además $div(1) = \{1,-1\}$
            \item $div(0) = \Z$
            \item $a \in div(n) \iff -a \in div(n)$
            \item $div(n) = div(-n)$
            \item $div(a) = div(b) \iff a = |b|$
        \end{enumerate}
    \end{definition}
        \begin{lemma}
            Si $b\neq 0$ y $a|b \Rightarrow |a|\leq |b|$
        \end{lemma}
        \begin{proof}
            $a|b \Rightarrow b = ak \Rightarrow |b| = |ak|=|a||k| \geq |a|$ esto último vale por que $|k| \in \Z$
        \end{proof}
        
        \begin{corollary}
            Sea $n\in \Z$ con $n\neq 0$ entonces $Div(n)$ es finito y más aún $Div(n) \subseteq [-n,n]$
        \end{corollary}    

    \section{Números Primos}

    Todo número entero $a$ tiene cuatro divisores triviales $\{a,-a,1,-1\}$.    
    Un número natural mayor que 1 es primo si tiene solo divisores triviales.

    \begin{remark} Veamos un par de detalles

        \begin{enumerate}
            \item 1 no es primo
            \item 2 es primo (-2 no)
            \item 0 no es primo
            \item si p y q son primo entonces o no se dividen o son el mismo primo ($p|q$ con p y q primos entonces $p=q$)  
        \end{enumerate}
    \end{remark}

    \begin{theorem}[Teorema fundamental de la aritmética]
        Todo numero natural mayor que 1 se puede escribir como producto único de primos salvo permutaciones de orden
        Además si el número es entero no natual menor que -1 se puede escribir como producto de primos multiplicados por -1
    \end{theorem}
    \begin{proof}
        Veamos la existencia usando inducción fuerte. el caso base es trivial por que es 2 que ya és primo

        Veamos la inducción, nuestra hipótesis es que todo numero $2\leq k \leq n$ se puede escribir como producto de primos

        Ahora tenemos $n+1$ si es primo ya está escrito como producto de primos, si no és primo tiene algún divisor no trivial j
        entonces $n+1=jl$ con $2<j<n+1 $ y $2<l<n+1$ por lo tanto $j$ y $l$ por hipótesis pueden ser escritos como productos de primos.

        Pero entonces $k$ puede ser escrito multiplicando j y l como producto de primosz


    \end{proof}

    \begin{proposition}
        Directamente del TFA tenemos que todo entero diferente de $\pm 1$ es divisible por un primo
    \end{proposition}
    \begin{remark}
        Habría que correjirlo para números negativos, pero es directo. Si tenemos un número negativo lo
        multiplicamos por -1, ahora usamos TFA y el primo divisor que obtenemos, nos sirve para el número
        negativo también por que $p|a \iff p|-a$
    \end{remark}

    \begin{theorem}
        Existen infinitos primos
    \end{theorem}
    \begin{proof}
        Supongamos que hay finitos primos si los multiplicamos todos tenemos $p = p_1p_2\ldots p_n$. Ahora $p+1$
        no es primo, por que es mas grande que $p_n$ que era el primo mas grande. Pero por el
        inciso anteriór sucede que entonces existe $p_i$ entre nuestros primos tal que $p_i |p+1$ además
        $p_i | p$ pero entonces $p_i |1$ lo que es absurdo
    \end{proof}

    \begin{lemma}
        Dados dos enteros positivos a,b existen k y r con $0\leq r \leq b $ tales que $a = bk + r $. Mas aún 
        $q$ y $r$ son únicos. 
    \end{lemma}
    \begin{proof}
        Sea $H\{n\in\N | a <nb \}$ sabemos que $H\neq \emptyset$ entonces tiene primer elemento

        sea $n$ su primer elemento y $q = n-1$ entonces $qb\leq a<nb = (q+1)b$, si nó $qb \leq a$ por que si fuera $a < qb = (n-1)b$ entonces
        $n-1 \in H$ por lo cual $n$ no sería primer elemento

        Ahora tenemos que $0 \leq a -qb < b $ llamando $a-qb = r$ tenemos que $0\leq r < b$ y $a = bq + r$

        Veamos que son únicos, supongamos $bq + r = bq' + r'$ entonces $b(q-q') = r' -r$. sin pérdida de generalidades supongamos $r' > r$
        entonces $r' -r \geq 0$ y además dado que $r'<b $ por definición entonces $r'-r < b$.

        Pero entonces tengo que $b$ por un entero es igual a algo menor que $b$ y mayor igual que 0, la única forma de que 
        esto suceda es que dicho algo sea cero y $b$ por un entero también sea cero.

        Entonces $r'-r = 0 $ mostrando que $r=r'$ y $b(q-q') = 0$ como $b\neq 0$ entonces $q=q'$
    \end{proof}

    \begin{corollary}[Algoritmo de la división]
        Dados $a,b \in \Z$ con $b\neq 0$ existen únicos enteros $q$ y $r$ tales que 
        $a = bq + r$ con $0 \leq r < |b|$. 
        
        Notación: dicho $r$ se nota $r_b(a)$ que sería resto de dividir a por b
    \end{corollary}

    \begin{proof}
        Veamos por casos $a>0 \land b>0$ es lo mismo que la anteriór y $b = |b|$ entonces ya está demostra
        $a<0 y b>0$ tenemos que $-a>0$ entonces por inciso anteriór $-a = bk + r$ entonces $$a = -bk -r = -bk -r -b +b = b(-k-1) -r+b$$

        Ahora sabemos si llamamos $b-r = r' \land -k-1=q$ tenemos $a = bq +r'$ con  $0 \leq r' < b = |b|$. 

        La unicidad vale sale de la misma forma que el inciso anteriór

        Caso $a>0 \land b<0$ entonces $-b>0$ entonces $a = -bj + r' $ si tomamos $-j = q$ tenemos $a=bq +r'$ con $0\leq r' < -b < |b|$

        La unicidad sale igual

        Caso $a<0 \land b<0$ entonces $-a>0 \land -b>0$ por lo tanto $-a = -bk +r'$ entonces 
        $a = bk -r' = bk -b +b -r' = b(k-1) -r' - b = bq + r$ con $0 \leq r< |b|$
    \end{proof}
    
    \section{Máximo común divisór}    

    \begin{definition}
        El máximo común divisór (MCD) entre dos enteros a y b es valga la rebundancia el mas grande de 
        los divisores comunes entre a y b. Notación $(a:b)$

        Otra forma de verlo es $max(div(a) \cap div(b))$
    \end{definition}

    \begin{remark}
        el MCD está bien definido por que $div(a) \cap div(b)$ seguro es no vacío, dado que el 1 pertenece
        a ambos conjuntos y son finitos ambos siempre y cuando $a \neq 0 \neq b$ por lo tanto la intersección
        es finita también. Por otro lado todos los divisores de a y b son enteros,entonces su intersección
        seran enteros
    \end{remark}

    \begin{remark}
        Propiedades
        \begin{enumerate}
            \item $(a:b) \geq 1$
            \item $(a:b) = (b:a)$
            \item $(1:a) = 1\quad \forall a \in \Z$
            \item $(a:b) = (-a:b) = (a:-b)=(-a:-b) = (|a|:|b|)$
            \item Dos números a,b se dicen coprimos si $(a:b = 1)$
        \end{enumerate}
    \end{remark}

    \subsection{Combinaciones Lineales}
    \begin{definition}
        $(a:b)$ tiene como propiedad relevante que puede ser escrito como combinación lineal de 
        a y b, es decir $$\exists k,j\in\Z \ \text{ tal que } \ (a:b) = aj + bk$$
    \end{definition}

    \begin{lemma}[Caracterización I del MCD]
        Dados a y b no nulos $(a:b)$ es el menor natural que puede ser escrito como combinación lineal
        entre a y b.
    \end{lemma}
    \begin{proof}
        Sea $e = na + mb$ el menór natural que puede ser escrito como combinación lineal entre a y b.
        Veamos primero que divide a $b$ y a $a$.
        Supongamos que no divide a b entonces $b = qe + r$ con $0<r\leq |e| = e$. Luego $b = qna +qmb +r$
        por lo tanto $r = b -qna -qmb = b(1-qm) + (-qn)a $, pero entonces $r$ es una combinación lineal 
        de a y b mas chica que $e$ lo cual es absurdo
        Lo mismo sucede con a, entonces $e$ es divisr común, veamos ahora que es el más grande
        Tomemos otro divisor común $d$ ahora $d|a \Rightarrow d|na$ y $d|b \Rightarrow d|mb$ entonces $d|nm+mb = e$
        
        Por lo tanto $d \leq |e| = e$. Entonces cualquier otro divisor común es mas peqeño que $e$
    \end{proof}

    \begin{corollary}
        \begin{enumerate}
            \item Si $a|bc \land (a:c) = 1 \Rightarrow a|b$
            \item Si $a|c \land b|c \land (a:b)=1 \Rightarrow ab|c$
        \end{enumerate}
    \end{corollary}
    \begin{proof}
        \begin{enumerate}
            \item Como $(a:c) = 1$ entonces existen $n,m\in\Z$ tales que $1 = am + cn$ entonces $b = bam +bcn$
            
        \indent Ahora $a|bc \land a|bam \Rightarrow a | bam + bcn = b$
            \item $am +bn = 1$ entonces $cam +cbn = c$ pero $a|c$ entonces $ab|cb$ y $b|c$ entoces $ab|ca$.
            
            Juntando todo tenemos que $ab | cbn + cam = c$
        \end{enumerate}
        
    \end{proof}

    \begin{corollary}
        Dados $a,b\in\Z$ $\frac{a}{(a:b)}$, $\frac{b}{(a:b)}$ son coprimos
    \end{corollary}
    \begin{proof}
        Sea $(a:b) = am + bn$ entonces $1 = \frac{am}{(a:b)} + \frac{bn}{(a:b)} = 1 = m\frac{a}{(a:b)} + n\frac{b}{(a:b)}$
        
        Lo que implica que $1=(\frac{a}{(a:b)}:\frac{b}{(a:b)})$, por que no hay combinación lineal posible
        mas pequéna que 1
    \end{proof}

    \begin{lemma}
        Sea $a,b \in \Z$ entonces $(a:b) = (a:b+ma)$ para cualquier $n\in\Z$
    \end{lemma}
    \begin{proof}
        $(a:b) = aj +bk = aj +bk -kma + kma = (j-km)a + k(b +ma)$. Entoces escribi a $a$ y a $b+ma$ como
        combinación lineal.
        Supongamos que dicha combinación lineal no es la mas chica, entonces existe $k',j'\in \Z$ 
        tal que $ak' + (b+ma)j' < (j-km)a + k(b +ma) = (a:b)$ entonces reordenando $a(k' +mj) + bj' < (a:b)$
        Pero esto es absurdo, no podemos tener una combinación de $a$ y $b$ mas chica que $(a:b)$
    \end{proof}

    \begin{lemma}[Caracterización 2 del MCD]
        Sean $a,b$ y sea $d\in \N$ tal que 
        \begin{enumerate}
            \item $d|a \land d|b$
            \item $d'|a \land d'|b \Rightarrow d'|d$
        \end{enumerate}

        Entonces $d = (a:b)$
        
    \end{lemma}

    \begin{remark}
        Vale la recíproca también, si $d(a:b)$ entonces cumple 1 y 2
    \end{remark}

    \begin{corollary}
        Sean $a,b\in\Z$ con $a\geq b$ sabemos que existen $q,r$ tales que $a = qb + r$ con $0 \leq r < b$.
        Entonces $(a:b) = (b:r)$
    \end{corollary}
    \begin{proof}
        $(a:b) = (b:a) = (b: a + mb)$ ahora si tomamos $q= -m$ tenemos $(b:a+mb) = (b:a -qb)$ pero 
        $a-qb = r_b(a)$ entonces $$ (a:b) = (b:a) = (b: a + mb) = (b:a -qb) = (b:r_b(a)) $$    
    \end{proof}

    \begin{proposition}
        Usando este último resultado podemos tenemos una forma de calcular $(a:b)$ llamada algoritmo de euclides

        $(a:b) = (b:r_1) = (r_1:r_2) = \ldots = (r_n:0)$ con $|b| > r_1 > r_2 > \ldots > 0 $

        Como son todos números naturales esta suceción debe ser finita y nos quedamos. Finalmente $(a:b) = r_{n-1}$
    \end{proposition}

    \section{Mínimo Común Múltiplo}
    \begin{definition}
        Dados dos enteros no nulos a y b el mínimo común multiplo es el primer elemento de $div(a) \cap div(b)$. Notación [a:b] (mcd)
    \end{definition}

    \begin{proposition}
        Algunas propiedades
        \begin{enumerate}
            \item $[a:b] = [b:a] $
            \item $[1:b] = |b|$
            \item $[0,b]=0$
            \item $[a:b]=[|a|:|b|]$
            \item Si $k$ es un múltiplo de a y b entonces $[a:b]|k$
        \end{enumerate}
    \end{proposition}
    \begin{proof}
        Demostremos el último item. k es múltiplo de a y b por lo tanto es mas grande que $[a:b]$ que es el mínimo común múltiplo

        Ahora tenemos que $k = [a:b]q + r$ con $0\leq r < [a:b]$ pero $a|k \land b|k \land a|[a:b] \land b|[a:b]$ entonces $a|r \land b|r$

        Lo que nos dice que r es un común múltiplo, pero entonces $r\geq [a:b] \lor r=0$ finalmente $r=0$ mostrando que $[a:b]|k$ 
    \end{proof}

    \begin{definition}
        Caracterización del mínimo común múltiplo
        \begin{enumerate}
            \item $k$ es múltiplo de a y b
            \item sea $k'$ múltiplo de a y b entonces $k|k'$
        
        \end{enumerate}
        si y solo si $k=[a:b]$
    \end{definition}

    \begin{proposition}
        Sean $a,b$ dos naturales (uno de ellos puede ser 0) entonces $ab=(a:b)[a:b]$
    \end{proposition}
    \begin{proof}
        Análogamente queremos ver $\frac{ab}{(a:b)} = [a:b] $. Para empezar $\frac{ab}{(a:b)}$ es múltiplo común de a y de por que
        $\frac{ab}{(a:b)} = b\frac{a}{(a:b)}$ mostrando que es múltiplo de b y análogamente probamos que es múltiplo de a.
        Restaría ver que divide a todo otro mcd.
        sea $m$ un múltiplo común de a y b entonces $m=ra=kb$ entonces $\frac{ra}{(a:b)} = \frac{kb}{(a:b)}$ ahora tenemos que $\frac{ra}{(a:b)} = k\frac{b}{(a:b)}$
        por lo tanto $\frac{b}{(a:b)} | r\frac{a}{(a:b)}$ pero sabemos que $\frac{a}{(a:b)}, \frac{b}{(a:b)}$ son coprimos
        entonces $\frac{b}{(a:b)} | r$ entonces puedo escribir $r = \ell\frac{b}{(a:b)} $
        
        Finalmente $m = ra = \ell\frac{b}{(a:b)} a = \ell\frac{ab}{(a:b)} $ que es obviamente divisible por $\frac{ab}{(a:b)}$
    \end{proof}

    \begin{proposition}
        Sea $a = p_1^{k_1}p_2^{k_2}\ldots p_n^{k_n}$ y $b = p_1^{j_1}p_2^{j_2}\ldots p_n^{j_n}$ números expresados como factores primos
        entonces $a|b \iff k_i\leq j_i \quad \forall i\leq n$
    \end{proposition}

    \begin{proof}
        Ida, como $a|b$ entonces $b=ca$ como son iguales ambos lados tienen los mismos primos a las mismas potencias, ahora si $c=1 \iff a=b$
        pero el caso interesante es si $a\neq b$ esto implica que $c$ o bien es primo o se escribe como producto de primos, en ambos
        casos a no puede tener todos los primos a las potencias iguales que los de b, por que si no $ca\neq b$ entonces debe tener 
        alguna potencia menor

        La vuelta. Si $k_i \leq j_i \quad \forall i\leq n$ entonces basta con usar un $c$ que corrija y emparece todos los exponentes
        para que $b=ac$ dicho c sería $c= p_1^{j_1-k_1}p_2^{j_2k_2}\ldots p_n^{j_n-k_n} $
    \end{proof}

    \begin{corollary}
        Sale directo de la proposicón anteriór 
        Si $a=p_1^{k_1}p_2^{k_2}\ldots p_n^{k_n}$ entonces 
        $div(a) = \{\pm p_1^{j_1}p_2^{j_2}\ldots p_n^{j_n} : j_i \leq k_i \quad \forall i \leq n\}$
    \end{corollary}

    \begin{corollary}
        Si $a = p_1^{k_1}p_2^{k_2}\ldots p_n^{k_n}$ entonces a tiene $(k_1 + 1)(k_2 + 1)\ldots(k_n+1)$ divisores
    \end{corollary}

    \begin{lemma}
        Sea $a = p_1^{k_1}p_2^{k_2}\ldots p_n^{k_n}$ y $b = p_1^{j_1}p_2^{j_2}\ldots p_n^{j_n}$ entonces:
        \begin{enumerate}
            \item $(a:b) = p_1^{min(j_1,k_1)}p_2^{min(j_2,k_2)}\ldots p_n^{min(j_n,k_n)} $
            \item $[a:b] = p_1^{max(j_1,k_1)}p_2^{max(j_2,k_2)}\ldots p_n^{max(j_n,k_n)} $
        \end{enumerate}
    \end{lemma}

    \begin{corollary}
        Si $a|c \land b|c$ y $(a:b)=1$ entonces $ab|c$
    \end{corollary}

    \begin{proof}
        $(a:b) = 1 $ entonces existen $m,n\in\Z$ tal que $1 = ma + nb$ entonces $c = mac + nbc$. Por otro lado sabemos que $a|c$ entonces 
        $ab|cb$ entonces $ab|cbn$ también $b|c$ entonces $ba|ca \Rightarrow  ab|cam$ entonces $ab|mac + nbc = c$
    \end{proof}

    \section{Desarrollos P-Ádicos}
    \begin{proposition}
        Dado cualquier $n\in\N $ y dado $s \in \N$ con $s\geq 2$, para $n\in\N$ tenemos que $\exists ! \ a_0,a_1,\cdots,a_k$ con $0\leq a_i < s$ 
        tales que $$n = a_ks^k + a_{k-1}s^k-1+\cdots + a_1s + a_0$$.

        En estos casos $s$ es llamada la base. Notamos $n =(a_ka_{k-1}\ldots a_0)_s $
    \end{proposition}
    \begin{proof}
        Unicidad: Supongamos $n =b_ks^k + b_{k-1}s^k-1+\cdots + b_1s + b_0 = a_ks^k + a_{k-1}s^k-1+\ldots + a_1s + a_0$
        Entonces $$n = s(a_ks^{k-1} + a_{k-1}s^k-2+\cdots + a_1) + a_0  = sq_1 + a_0 \text{ con } 0\leq a_0 < s$$
        por otro lado $$ n = s(b_ks^{k-1} + b_{k-2}s^k-1+\cdots + b_1) + b_0 = sq_2 + b_0 \text{ con } 0\leq b_0 < s $$

        Entonces $sq_1 + a_0 = sq_2 + b_0$ pero por algoritmo de división $q_2,q_1,b_0,a_0$ 
        son únicos entonces $$q_1=q_2 \land b_0 = a_0$$ Mostrando que ambas formas eran exactamente iguales

        Existencia: si $n<s$ entonces $n = a_0$ ya nos queda lo que queríamos con su notación $n = (a_0)_s$.

        Si $n>s$ entonces algoritmo de división tenemos $q_1, r_1\in\Z$ tales que $n = sq_1 +r_1$. Ahora si $q_1 < s$ 
        ya tenemos lo que necesitamos, nos queda $n = (q_1r_1)_s$.

        Si $q_1>s$ entonces devuelta por algoritmo de divisón existen $q_2,r_2\in \Z$ tales que $q_1 = q_2s + r_2$. Juntando todo
        nos queda $n = s(q_2s+r_2) + r_1 = q_2 s^2 + r_2s + r_1$ devuelta si $q_2 < s$ ya tenemos lo que necesitamos y $n = (q_2r_2r_1)_s$

        Notemos que $q_1 > q_2 > \cdots > q_n > 0$ (son estrictamente decrecientes por algoritmo de división y 
        son mayores que cero por que s y n son números naturales y por algoritmo de divisón (pensar)).

        Entonces en algún momento $q_n < s$ por que son todos naturales y no hay infinitos naturales entre dos números naturales
        
    \end{proof}

    \section{Congruencia de Enteros}
        \begin{definition}
            Dada $m\in\N$ y $a,b \in\Z$ decimos que $a$ es congruente con $b$ módulo $m$ si $m|a-b$. Lo escribimos $a\equiv b (m)$
        \end{definition}
        \begin{proposition}
            Es trivial notar que congruencia es una relación de equivalencia
        \end{proposition}

        \begin{proposition}
            Algunas propiedades
            \begin{enumerate}
                \item $a\equiv b (m)\land c \equiv d(m)$ entonces $a+c \equiv b+d (m) \land ac \equiv bd (m)$
                \item Si $a\equiv b(m) \land \alpha \equiv \beta (m)$ entonces:
                    \begin{enumerate}
                        \item $ax + \alpha y \equiv bx + \beta y $
                        \item $a^n \equiv b^n (m)\quad \forall n\in\N$
                        \item (Polinomios) $f(a) \equiv f(b)(m)\quad\forall f\in\Z[X]$
                    \end{enumerate}
            \end{enumerate}
        \end{proposition}

        \begin{proposition}
            $a \equiv b(m) \iff ac\equiv bc (mc)$
        \end{proposition}
        \begin{proof}
            $m|a-b \Rightarrow a-b=km \iff ca -cb = k(cm)$ finalmente $cm|ca-cb$ entonces $ca\equiv cb (mc)$
        \end{proof}

        \begin{proposition}
            Si $ac \equiv bc (m)$ y $d=(m:c)$ entonces $a\equiv b(\frac{m}{d})$
        \end{proposition}
        \begin{proof}
            $ac-bc=mk \iff \frac{c}{d}a - \frac{c}{d}b = k\frac{m}{d}\iff \frac{c}{d}(a-b) = k\frac{m}{d} $

            Entonces $\frac{m}{d} | \frac{c}{d}(a-b)$ como $\frac{m}{d}$ es coprimo con $\frac{c}{d}$ concluimos que
            $\frac{m}{d} | (a-b)$ o lo que es lo mismo $a\equiv b (\frac{m}{d})$
        \end{proof}

        \begin{corollary}
            Algunos resultados
            \begin{enumerate}
                \item Del la anteriór proposición se sigue $ac\equiv bc (mc) \iff a\equiv b (m)$
                \item Si $(m:c)=1$ entonces $ac\equiv bc (m) \iff a\equiv b (m)$ 
                \item Si $p$ primo y $p\neq m$ entonces $ap\equiv bp (m) \iff a\equiv b(m)$
            \end{enumerate}
        \end{corollary}

        \begin{remark}
            Para el segundo y tercer item: las vueltas valen siempre, pero las idas requieren $(m:c) = 1 \land p\neq m$ respectivamente
            si no se cumplen estamos en el caso mostrado en la proposición anteriór
        \end{remark}

        \begin{lemma}
            Si $a\equiv b(m)$ y $0\leq |b-a| <m$ entonces $a=b$
        \end{lemma}
        \begin{proof}
            $a-b=km \Rightarrow |a-b| = |k|m$ pero por hipótesis $0 \leq |a-b| < m$ entonces $k=0$ 
            necesariamente
        \end{proof}
            
        \begin{lemma}
            Dos enteros son congruentes módulo m si y sólo si tienen el mismo resto dividio por m
        \end{lemma}
        \begin{proof}
            Ida. Tenemos $a\equiv b (m)$ con $a=q_1m+r_1 \land b=q_2m +r_2$. Sabemos que $m|a-b$

            Equivalentemente $m|(q_1 - q_2)m +r_1-r_2$ y $m|(q_1-q_2)m$ entonces $m|r_1-r_2$ o lo mismo 
            $r_1\equiv r_2 (m)$
            
            .Pero además sabemos que $0<r_1 <m \land 0<r_2 <m$ entonces sabemos que $r_1 -r_2 < m $
            por lo tanto $0<|r_1-r_2|<|m| = m$ 

            Entonces $r_1 = r_2$ usando el lema anteriór

            Vuelta. Sabemos que $a = q_1m + r_1$ y $b=q_2m+r_2$ con $r_1=r_2$ entonces $a-b = (q_1 -q_2)m$

            Entonces $m|a-b$ mostrando que $a\equiv b (m)$
        \end{proof}

        \begin{proposition}
            Si $a\equiv b (m)$ y $a\equiv b (n)$ y $(m:n) = 1$ entonces $a\equiv b (mn)$
        \end{proposition}
        \begin{proof}
            $m|a-b \land n|a-b$ como $(m:n) =1$ entonces $mn|a-b$ mostrandoq ue $a\equiv b(mn)$
        \end{proof}
        \begin{remark}
            La recíproca también vale
        \end{remark}
        \begin{corollary}
            Sea $m=p_1^{j_1}p_2^{j_2}\ldots p_n^{j_n}$ sus factores primos entonces $a\equiv b(m)\iff a\equiv b(p_i^{j_i})$
        \end{corollary}

        \begin{corollary}[Sistemas equivalentes]
            $a_1\equiv b (c_1) \land a_2\equiv b (c_2) \ldots a_n\equiv b(c_n)$ con $c_1\ldots c_n$ coprimos
            2 a 2 $\iff a_1a_2\ldots a_n \equiv b (c_1c_2\ldots c_n)$ 
        \end{corollary}


        \begin{lemma}
            Sea $a\in K$ con $a\neq 0$ entonces existe un $a'\in\Z$ no nulo tq $aa' \equiv 1(m) \iff (a:m)=1$.
            Mas aún $a'$ es único congruente m
        \end{lemma}
        \begin{proof}
            Unicidad supongamos exsiten $a',a''\in\Z$ tal que $aa'\equiv 1(m) \land aa''\equiv 1(m)$

            Entonces $aa'\equiv aa''(m)$ como $(a:m)=1$ puedo dividir de ambos lados
            por lo tanto $a'\equiv a'' (m)$

            Ida. $m|aa'-1$ entonces si $d=(a:m)$ sabemos que $d|m$ por lo tanto $d|aa'-1$ pero además 
             $d|a$ y entonces $d|1$ y $d\geq 1$ entonces $d=1$
            
            Vuelta. Existen $k,p\in\Z$ tales que $1 = ka + pm$ entonces $ka-1 = -pm$ entonces $m|ka-1$
            o lo que es lo mismo $ka\equiv 1(m)$ ahora tomamos $k=a'$ y obtenemos lo que queríamos
        \end{proof}

        \begin{remark}
            Si $a\equiv b(m) \land (b:m)=1 \Rightarrow (a:m)=1$
        \end{remark}
        \begin{proof}
            $a=b+km$ como si $d|a \land d|m$ entonces $d|b$ luego $d$ es divisor común de b y m
            por lo tanto es $d=1$
        \end{proof}


        \section{Teorema de Euler}
        \begin{definition}[Phi de Euler]
            $\phi (n) = | \{a\in\Z \ | \ 1<a\leq n \land (a:n)=1 \}|$
            
        \end{definition}
        
        \begin{remark}
            Algunas propiedades. Si $p$ primo
            \begin{enumerate}
                \item $\phi(p) = p-1$
                \item $\phi(p^2) = p^2-p$
                \item $\phi(p^k) = p^k - p^{k-1}$
                \item $m,n \in \N$ entonces $\phi(n)\phi(m) = \phi(nm)$ si $(m:n)=1$
            \end{enumerate}
        \end{remark}
        \begin{definition}
            Una forma de calcular $\phi(m)$
            $$\phi(m)=m\prod_{p|m}^n (1-\frac{1}{p_i})$$
        \end{definition}
        \begin{definition}
            Un sistema residual completo (de ahora en mas llamado sistema residual) es un conjunto 
            con m elementos, cada uno un representante de cada una de las m clases de equivalencia módulo m

            Un sistema residual reducido es un subconjunto del anteriór donde solo estan los representantes
            coprimos con m
        \end{definition}

        \begin{lemma}
            Sea $(k:m)=1$
            \begin{enumerate}
                \item Si $\{a_1,a_2,\ldots a_n\}$ es un sistema residual 
                módulo m entonces $\{ka_1,ka_2,\ldots ka_n\}$ es un sistema residual
                \item Si $\{b_1,b_2,\ldots b_{\phi(m)}\}$ es un sistema reducido entonces 
                $\{kb_1,kb_2,\ldots kb_{\phi(m)}\}$ es un sistema reducido
            \end{enumerate}
        \end{lemma}

        \begin{proof}
            \begin{enumerate}
                \item Probemos que son todos no congruentes en $\{ka_1,ka_2,\ldots ka_n\}$.
                Supongamos que tenemos  $ka_i \equiv ka_j (m)$ en el conjunto, por un lado sabemos que 
                $a_i$ y $a_j$ deben estar en $\{a_1,a_2,\ldots a_n\}$ pero como $(k:m)=1$ 
                tenemos que $a_i \equiv a_j (m)$ pero no puedo tener dos número equivalentes en el mismo sistema
                por como se define sistema residual (un representante de cada clase de equivalencia)


                Entonces $\{ka_1,ka_2,\ldots ka_n\}$ tiene elementos NO equivalentes entre si y tiene la misma
                cantidad de elementos que $\{a_1,a_2,\ldots a_n\}$ mostrando que es sistema residual módulo m
                
                \item Sabemos,usando la misma idea que el ejericio anteriór que ningún par de 
                $k_i,k_j \in \{kb_1,kb_2,\ldots kb_n\}$ son equivalentes. Además ambos sistemas tiene la misma
                cantidad de elementos. 
                
                Por otro lado $$(b_i:m)=1 \quad\forall i\in\N \quad 1\leq i \leq \phi(m)$$ justamente por 
                definición de sistema reducido. Además, por hipótesis $(k:m)=1$ 
                entonces $$(kb_i:m)=1 \quad\forall i\in\N \quad  1\leq i \leq \phi(m)$$

                Luego $\{kb_1,kb_2,\ldots kb_{\phi(m)}\}$ es un sistema reducido
            \end{enumerate} 
        \end{proof}        


        \begin{theorem}[Teorema de Euler Fermat]
            Si $(a:m)=1$ entonces $$a^{\phi(m)} \equiv 1 (m)$$
        \end{theorem}

        \begin{proof}
            Sea $$\{b_1\ldots b_{\phi(m)}\}$$ un sistema reducido mod m entonces $$\{ab_1\ldots ab_{\phi(m)}\}$$
            También lo és por ser $(a:m)=1$

            Pero entonces $$b_1\ldots b_{\phi(m)} \equiv ab_1\ldots ab_{\phi(m)} (m) $$ Esto vale por que ambos
            conjuntos tienen representantes modulo m y los representantes son equivalentes mod m

            Entonces $$b_1\ldots b_{\phi(m)} \equiv a^{\phi(m)}(b_1\ldots b_{\phi(m)}) (m)$$ Sabemos que 
            todos los $b_i$ son coprimos con m por definición de sistema residual reducido, entonces 
            $$1  \equiv a^{\phi(m)} (m)$$ 
        \end{proof}

        \begin{corollary}[Pequeño teorema de Fermat]
            Si p primo entonces $\phi(p) = p-1$ por lo tanto tendríamos que $(p:a) = 1 $
            $$ a^{p-1} \equiv 1 (p)$$

            que es el pequeño teorema de fermat, como caso particular de Euler , por lo tanto ya quedó
            demostrado
        \end{corollary}

        \begin{theorem}[Teorema de Wilson]
            Si $p$ primo entonces $(p-1)! \equiv -1 (p)$
        \end{theorem}
        \begin{proof}
            queremos ver que $1.2\ldots (p-2)(p-1) \equiv -1 (p)$ sabemos que $p-1\equiv -1 (p)$
            entonces queremos ver que $1.2\ldots (p-2) \equiv 1 (p)$.

            Sabemos que si $1 \leq a \leq p-2$ entonces $(a:p)=1$ por lo tanto usando lema anteriór 
            tenemos que para cada $a$ existe un $a*$ tal que $aa*\equiv 1(p)$ y si $a_1\neq a_2$ 
            entonces $a_1*\neq a_2*$ por unicidad además es único congruente p por lo tanto
            tenemos un representante menor estricto que p

            ** Veamos que para cada factor a de $(p-2)!$ a es distinto de su inverso $a*$. Supongamos que no,
            entonces tenemos un $a\in [2,p-2]$ tal que $a=a*$ entonces $a^2 = aa* \equiv 1 (p)$. 
 
            Luego $(a-1)(a+1) = a^2 -1 \equiv 0 (p)$ entonces $a-1\equiv 0 (p)$ o $a+1\equiv 0 (p)$. Pero sabemos
            que $a-1 \land a+1$ son menores que p entonces $a=1$ o $a = p-1$ lo cual es absurdo en ambos casos.
            Además es facil ver que 1 es inverso de si mismo y $(p-1)(p-1)=p^2 -2p +1 \equiv 1 (p)$ entonces 
            $p-1$ es inverso de si mismo también.

            Pero entonces todos los otros inversos seran menores que $p-1$ y mayores que 1, por que ya dijimos
            no pueden ser inversos de dos numeros diferentes, entonces para $a$ tal que $1<a <p-1$, existe
            $a*$ inverso de a modulo p

            Entonces podemos reescribir y reordenar $(p-2)! = 1.(a_1a*_1)\ldots (a_qa*_q) \equiv 1 (p)$

            Entonces $(p-1)! = (p-1)(p-2)! \equiv (p-1) (p)$ y $p-1 \equiv -1 (p)$

            ** Esta es una demostración también de que $p-1 \land 1$ son inversos de si mismo modulo $p$. 
            Es por esto que no lo podemos simplificar y al resto si
        \end{proof}

        \begin{definition}
            Dado $n\in\N$ definimos $\Z_n$ como el conjunto de clases de equivalencia módulo m
            $Z_n = \{[0]_m,[1]_m,\ldots , [n-1]_m\} = \{[1]_m,[2]_m \ldots [m-1]_m\}$
        \end{definition}
        \begin{definition}
            Definimos 
            \begin{enumerate}
                \item $+:Z_n\times Z_n \rightarrow Z_n $ dada por $[a]+[b] = [a+b]$
                \item $*:Z_n\times Z_n \rightarrow Z_n $ dada por $ [a]*[b] = [a*b]$
            \end{enumerate}
        \end{definition}     
        
        \begin{definition}
            Usando esto se puede probar que $(Z_n,+,*)$ es un anillo conmutativ es un anillo conmutativoo
        \end{definition}


        \begin{definition}
            Sea $n\in\N$ y $[a]\in \Z_n$
            \begin{enumerate}
                \item $[a]$ es unidad si $\exists [b]$ tq $[a]* [b] = [1]$
                \item $[a]$ es un divisor de cero si $\exists [b]\in \Z_n $ diferente de $[0]$ 
                tq $[a]* [b]=0$
            \end{enumerate}
        \end{definition}

        \begin{proposition}
            Algunas propiedades
            \begin{enumerate}
                \item $[n-1]$ es unidad modulo n
                \item Si $a|m$ entonces $[a]$ es divisor de cero módulo n
                \item Si $(a:m) > 1$ entonces $[a]$ es divisor de cero módulo n
                \item $[a]$ es unidad módulo m $iff$ $(a:m)=1$, ya lo habíamos visto
            \end{enumerate} 

        \end{proposition}
        \begin{corollary}
            $\Z_n$ no tiene divisor de cero o equivalentemente todos sus elementos no nulos son unidades
            (son invertibles) si y sólo si $m$ es primo. (Otra forma de decirlo $\Z_p$ es un cuerpo si 
            y sólo si p es primo)
        \end{corollary}

        \section{Ecuaciones lineales de congruencia}

        \begin{definition}
            Una ecuación lineal de congruencia tiene esta pinta $$ ax \equiv b (m)$$
            y tiene solución si y sólo si $(a:m) | b$. Si tiene solución tiene infinitas, aunque ya veremos
            que módulo m no son infinitas
        \end{definition}
        \begin{theorem}
            Consideremos $ax\equiv b (m)$
            \begin{enumerate}
                \item $(a:m) = 1$ entonces la ecuacíon tiene solución y es única módulo m
                \item $(a:m) =d\neq 1$ entonces la ecuación tiene solución si sólo si $d|b$. Y dicha solución
                es única módulo $\frac{m}{d}$
            \end{enumerate}
        \end{theorem}
        \begin{proof}
            Tenemos $ax\equiv b(m)$
            \begin{enumerate}
                \item $(a:m)=1$ entonces tenemos $ak +mj = 1$ por lo tanto $akb + mjb = b$ entonces
                si tomamos $jb=q \land kb = x'$ tenemos $ax' - b = qm$ entonces tenemos que $x'$ cumple
                la ecuación.

                Veamos que es única módulo m. Supongamos que tenemos dos soluciones distintas $c \land c'$ 
                modulo m entonces $c<m \land c'<m$. Pero $ac \equiv b \equiv ac' (m)$ entonces $ac\equiv ac' (m)$
                como $(a:m)=1$ tenemos $c\equiv c'(m)$ como ambas son menores que m entonces $c=c'$ absurdo

                \item Ida la ecuación tiene solución entonces $ax -b = km$ ahora como $d=(a:m)$ entonces 
                $d|-ax \land d|km$ entonces $d|b$. 

                La vuelta $d|b$ y $(a:m) = 1$ entonces $ax\equiv b(m) $ tiene solución si y sólo si
                existe $k\in\Z$ tal que $mk=ax-b$ si y sólo si $k\frac{m}{d} = \frac{a}{d}x - \frac{b}{d}$
                si y sólo si $\frac{a}{d}x \equiv \frac{b}{d} (\frac{m}{d})$, pero sabemos que 
                $(\frac{a}{d} :\frac{m}{d}) = 1$ entonces estamos en el caso 1, por lo tanto tiene solucíon
                Como la última ecuación tiene solucíon entonces por la cadena $ax\equiv b (m)$ también tiene.

                Veamos que es única sean $x_0,x_1$ soluciones, entonces $ax_0\equiv b(m)$ y $ax_1\equiv b(m)$
                usando la misma cadena que el párrafo anteriór llegamos a 
                $\frac{a}{d}x_0 \equiv \frac{b}{d} (\frac{m}{d})$ y 
                $\frac{a}{d}x_1 \equiv \frac{b}{d} (\frac{m}{d})$ 
                entonces $\frac{a}{d}x_0 \equiv \frac{a}{d}x_1 (\frac{m}{d})$ como 
                $(\frac{a}{d}:\frac{m}{d})=1$ entonces $x_0\equiv x_1 (\frac{m}{d})$. Mostrando que la solución
                es única modulo $\frac{m}{d}$
            \end{enumerate}
        \end{proof}

        \begin{proposition}
            El sistema $x\equiv b_1 (n_1)$ y $x\equiv b_2 (n_2)$ tiene solución si y sólo si $(n_1:n_2)|b_1 -b_2$

            Además la solución es única mod $[n_1:n_2]$
        \end{proposition}
        \begin{proof}
            Ida: Supongamos $x_0$ es solución entonces $n_1 | x_0 - b_1$ y $n_2 |x_0 -b_2$ entonces 
            $(n_1:n_2)|x_0-b_2 - (x_0 -b_1) = b_1 - b_2$

            Vuelta: $(n_1:n_2) |b_1-b_2$ entonces $n_1x \equiv b_1-b_2 (n_2)$ tiene solución
            
            Por lo tanto existe $x_0$ tal que $n_1x_0 = b_1-b_2 + n_2l$. Pero entonces tomo 
            $k_0 = n_2l -b_2 =n_1x_0 -b_1 $ y $k_0$ satisface $k_0 \equiv b_2 (n_2)$ y $k_0\equiv b_1(n_1)$
            Mostrando que dicho sistema tiene solución

            Finalmente si tenemos dos soluciones $x_0,x_1$ entonces $x_0 - x_1 \equiv 0 (n_1)$
            y $x_0 - x_1 \equiv 0 (n_2)$. Entonces $n_1|x_0-x_1$ y $n_2|x_0-x_1$ entonces $x_0-x_1$
            es común múltiplo de $n_1$ y $n_2$ por lo tanto $[n_1:n_2]|x_0-x_1$ 
            
            Equivalentemente
            $x_0\equiv x_1 ([n_1:n_2])$
        \end{proof}

        \begin{theorem}[Teorema Chino del Resto]
            Sean $n_1,n_2\ldots n_k \in \N$ coprimos dos a dos el sistema de congruencia $x\equiv b_1(n_1)$
            \[ 
                \begin{cases} 
                x\equiv b_1 (n_1) \\
                x\equiv b_2 (n_2) \\
                 \vdots \\
                x\equiv b_k (n_k)
                \end{cases}
            \]
                Tiene solución y es única mod $\prod_{i=1}^k n_i$
        \end{theorem}

        \begin{proof}
                Sea $n = n_1n_2\ldots n_k$ y para cada $i\in\N$ tal que $1\leq i \leq k$ sea 
                $a_i = \frac{n}{n_i}$ como $(n:a_i) = 1$ tenemos que $a_ix \equiv b_1 (n)$ tiene solución
                llamemos a dicha solución $x_i$ ahora si tomamos $z = \sum_{i=1}^{k} a_ix_i$ tenemos que 
                $z \equiv a_ix_i (n_i)$ (por que $n_i$ va a dividir a todos los $a_j$ tal que $j\neq i $)
                entonces $z \equiv b_i (n_i)$ y esto vale para cualquier $i\in\N$ tal que $0\leq i \leq k$
                Por lo tanto $z$ es solución de todas las ecuaciones del sistema.

                Veamos que es unica mod $\prod_{i=1}^k n_i$. Sea $x_0$ y $x_1$ dos soluciones, entonces
                $x_0\equiv b_i (n_i)$ y $x_1\equiv b_i(n_i)$ para todo $i\in\N$ con $0\leq i \leq k$
                Entonces $x_0\equiv x_1 (n_i)$ para todo $i\in\N$ con $0\leq i \leq k$
                Entonces $n_i| x_0 - x_1$ para todo $i\in\N$ con $0\leq i \leq k$, como son todos coprimos
                $n_1n_2\ldots n_k | x_0 - x_1$ lo que me dice que $x_1 \equiv x_0 (n_1n_2\ldots n_k)$
        \end{proof}

        \newpage
        \section{Números Reales}
        \begin{definition}[Axiomas de cuerpo]
            Teniendo un conjunto $A$ y las dos operaciones típica $(+,*)$
            Entonces si $(A,+,*)$ es un cuerpo cumple
            Sobre la suma \begin{enumerate}
                \item Asociativa
                \item Conmutativa
                \item Tiene elemento neutro
                \item Existe opuesto
            \end{enumerate}
            Sobre la multiplicación
            \begin{enumerate}
                \item Asociativa
                \item Conmutativa
                \item Tiene elemento neutro
                \item Existe opuesto
            \end{enumerate}
            Sobre la suma y la multiplicación, la ley de la distributiva
        \end{definition}

        \begin{remark}
            Todo cuerpo es un anillo en particular, $\R$ es un cuerpo
        \end{remark}

        \begin{definition}[Axiomas de cuerpo ordenado]
            Todos los del cuerpo y estas otras
            \begin{enumerate}
                \item Tricotomia y transitividad
                \item Compatibilidad en la suma: $a\geq b \Rightarrow a+c \geq b+c$
                \item Compatibilidad en la multiplicación: $c>0 \land a>b \Rightarrow ca> cb$
            \end{enumerate}
        \end{definition}
        \begin{remark}
            $\R$ resulta un cuerpo ordenado (dándole el órden tradicional). Pero $\Z_p$ es un cuerpo no ordenado
            $\mathbb{C}$ tampoco
        \end{remark}

        \begin{definition}[Axiomas de cuerpos ordenados y completos]
            Tiene todos los del cuerpo ordenado y además cumple el axioma de completitud

                Axioma de completitud: Todo conjunto no vacío y acotado superiormente tiene supremo
        \end{definition}
        \begin{remark}
            $\R$ es completo, pero $\mathbb{Q}$ no lo és
        \end{remark}

        \begin{proposition}
            En todo cuerpo (en particular en $\R$) valen 
            \begin{enumerate}
                \item $x0=0$
                \item $x+y = x+z \Rightarrow y=z$
                \item $\exists \ !x$ tq $a+x=b$
                \item $ab=0 \Rightarrow a=0 \lor b=0$
                \item $-(-x)=x$
                \item $-(x+y)=(-x)+(-y)$
                \item $=-(xy)=(-x).y = x.(-y)$
                \item $(-x)(-y) = xy$
            \end{enumerate}
        \end{proposition}
        Todas salen usando que un cuerpo es anillo conmutativo

        \begin{proposition}
            En todo cuerpo valen
            \begin{enumerate}
                \item $(x^{-1})^{-1}=x$
                \item $(xy)^{-1} = x^{-1}y^{-1}$ 
                \item $(-1)^{-1}=-1$
                \item $(-x)^{-1} = -x^{-1}$
            \end{enumerate}
        \end{proposition}
        \begin{proof}
            Veamos cada una
            \begin{enumerate}
                \item $(x^{-1})^{-1}=x \iff (x^{-1})^{-1}x^{-1}=1$ y el de la derecha vale por definición
                \item tenemos que $(xy)^{-1} = x^{-1}y^{-1} \iff (xy)^{-1} (xy)=1$.
                Entonces $(xy)^{-1}$ es inverso de $xy$ entonces $(xy)^{-1} = x^{-1}y^{-1}$
            \end{enumerate}
        \end{proof}

       \begin{proposition}
            En todo cuerpo vale 
            \begin{enumerate}
                \item $xy=0 \Rightarrow x=0 \lor y=0$
                \item En particular $xy=\neq 0 \Rightarrow x\neq 0 \lor y\neq0$
                En particular 
                \begin{enumerate}
                    \item $x^2 =0 \iff x=0$
                    \item $x^2 =1 \iff x=1 \lor x=-1$
                    \item $x^{-1}=x \iff x=1 \lor x=-1$
                \end{enumerate}
            \end{enumerate}
        \end{proposition}
        \begin{proof}
            Demostremoslas
            \begin{enumerate}
                \item Ya lo vimos arriba
                \item Supongamos $x\neq 0$ entonces como $xy=0$ sucede $x^{-1}xy=0$ por lo tanto $y=0$
                \begin{enumerate}
                    \item $x^2=0 \Rightarrow x.x=0 \Rightarrow x=0 \lor x=0$ usando la de arriba
                    \item $x^2=1 \iff x^2-1=0 \iff (x-1)(x+1) = 0 \iff x-1=0 \lor x+1=0$
                    \item $x^{-1}=x \iff x^2=1 \iff  x=1 \lor x=-1$ usando $b$
                \end{enumerate}

            \end{enumerate}
        \end{proof}

        \begin{proposition}
            En todo cuerpo ordenado valen
            \begin{enumerate}
                \item $0<1$
                \item $x>0 \iff -x<0$
                \item $x<y \iff -x>-y$
                \item $x<y \land m < b \Rightarrow x+m>y +b$
                \item $x<y \land z<0 \Rightarrow xz > yz$
                \item
                \begin{enumerate}
                    \item $x>0 \land y>0 \Rightarrow xy>0$
                    \item $x>0 \land y<0 \Rightarrow xy<0$
                    \item $x<0 \land y<0 \Rightarrow xy>0$
                    \item $x<0 \land y>0 \Rightarrow xy<0$
                \end{enumerate}
                \item $x>0 \iff x^{-1} >0$
            \end{enumerate}
        \end{proposition}
        \begin{proof}
            \begin{enumerate}
                \item Trivial
                \item $x>0 \iff -x +x > -x +0 \iff 0> -x$
                \item $x<y \iff -x - y + x < -x -y +y \iff -y < -x $
                \item $x + m < y +m < y+b$
                \item $x<y$ por 2 $-z>0$ entonces $-zx<-zy$ por 3 $xz>yz$
                \item Salen todas con el 5.
                \item $x>0$ suponemos $x^{-1} < 0$ entonces $xx^{-1} < 0x^{-1} \iff 1 < 0$ abs 
            \end{enumerate}
        \end{proof}


        \begin{proposition}
            En todo cuerpo ordenado valen:
            \begin{enumerate}
                \item $x^2 \geq 0 \quad \forall x$ en dicho cuerpo y $x^2 = 0 \iff x=0$
                \item $x^2 + y^2 \iff x=0 \land y=0$
                \item $0<x<y \Rightarrow x^2<y^2$
                \item $x<y<0 \Rightarrow x^2>y^2$
                \item $0<x<y\Rightarrow y^{-1} < x^{-1}$
                \item $x<0<y \Rightarrow x^{-1} < y^{-1}$
                \item $x<y<0 \Rightarrow y^{-1} < x^{-1}$
            \end{enumerate}
        \end{proposition}
        \begin{proof}
            Veamos las demostraciones
            \begin{enumerate}
                \item Se sigue de 6 de la prop anteriór y la segunda parte ya la habíamos probado
                \item Si $x^2 = -y^2$ pero $x^2\geq  0 \land y^2 \geq 0$ entonces $y=0=x$ por que si $y>0$
                entoncse $y^2 > 0$ entonces $-y^2 <0$ entonces $x^2 = -y^2 <0$ que es absurdo
                \item Por hipótesis $xx < yx < yy$
                \item Como $x<0 $ tenemos $xx>yx$ como $y<0$ tenemos $xy> yy$ entonces $x^2 >y^2$
                \item $x>0 \land y>0 \Rightarrow 1 < yx^{-1} \Rightarrow y^{-1} < x^{-1}$
                \item $x<0 \Rightarrow 1 > x^{-1}y \Rightarrow y^{-1} > x^{-1}$
                \item Sale con la misma idea que los otros dos 
            \end{enumerate}
        \end{proof}

        \section{Números Complejos}

    \end{document}
