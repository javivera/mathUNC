
\documentclass[10pt]{extarticle}
\usepackage{amssymb}
\usepackage{amsmath}
\usepackage{amsthm}
\usepackage{mathpazo}
\usepackage{tcolorbox}
\usepackage[margin=0.8in]{geometry}
\usepackage[colorlinks=true]{hyperref}
\usepackage{tcolorbox}
\usepackage[shortlabels]{enumitem}

\newtheoremstyle{break}
{\topsep}{\topsep}%
{\itshape}{}%
{\bfseries}{}%
{\newline}{}%
\theoremstyle{break}
\newtheorem{theorem}{Teorema}[section]
\newtheorem{corollary}{Corolario}[theorem]
\newtheorem{lemma}[theorem]{Lema}
\newtheorem{proposition}{Proposición}
\newtheorem*{remark}{Observación}
\newtheorem{definition}{Definición}[section]
\theoremstyle{definition}
\newtheorem{example}{Ejemplo}[section]
\newcommand\quotient[2]{{^{\displaystyle #1}}/{_{\displaystyle #2}}}
\synctex=1
\renewcommand{\labelenumii}{\arabic{enumi}.\arabic{enumii}}
\renewcommand{\labelenumiii}{\arabic{enumi}.\arabic{enumii}.\arabic{enumiii}}
\renewcommand{\labelenumiv}{\arabic{enumi}.\arabic{enumii}.\arabic{enumiii}.\arabic{enumiv}}

\begin{document}

\title{Teorico Estructuras Algebraicas}
\author{Javier Vera}
\maketitle


\section{Clase 1}
\section{Clase 2}
\section{Clase 3}
\section{Clase 4}
\section{Clase 5}
\section{Clase 6}
\section{Clase 7}
\section{Clase 8}
\section{Clase 9}
\section{Clase 10}
\section{Clase 11}

\section{Clase 12}

\begin{definition}[Accion de Grupo]
	Sean $G$ grupo y $X\neq\emptyset$ conjunto. Una accion de $G$ en $X$ es una funcion
	\begin{align} 
	& G\times X\longrightarrow X \nonumber\\
	& (g,x)\longmapsto g.x \nonumber
	\end{align}
	Que cumple:

	\begin{enumerate}
		\item $gh.x=g.(h.x)$
		\item $e.x=x\quad\forall x\in X$
	\end{enumerate}
En este caso se dice que $G$ actua (opera) en $X$ mediante $G\times X\longrightarrow X$
\end{definition}

\begin{example}
	\begin{enumerate}
		\item $G,X\neq\emptyset$ cualesquiera la accion trivial de $G$ en $X$ es aquella tal que $g.x=x\quad\forall x\in X\quad\forall g\in G$
	  	\item $\mathbb{S}(x)$ actua en $X$ en la forma $\mathbb{S}\times X\longrightarrow X$ $\sigma.x=\sigma(x)\quad\forall \sigma\in \mathbb{S}(x)\quad\forall x\in X$. En particular $\mathbb{S}$ actua en $I_{n}=\{ 1,\ldots n \}$
		\item Sea $G$ grupo actua en si mismo de distintas formas, en este caso mediante el producto $G\times G \longrightarrow G$ es decir $g.x=gx$ esto se llama *accion regular*
		\item $H\trianglelefteq G$ entonces $G$ actua por conjugacion $G\times H \longrightarrow H$ dada por $g\in G\quad x\in H$
		\item $\mathbb{S}(G)=\{ \text{subgrupos de } G \}$. entonces $G$ actua en $\mathbb{S}$ por conjugacion $g\in G\quad H\trianglelefteq G$
		\item  $H\leq G$ entonces $G$ actua en las coclases $G\diagup H$
	\end{enumerate}
	Ejercicio probar que satisfacen (A1) y (A2)
\end{example}

\begin{proposition}
	Sea $G$ grupo $X\neq\emptyset$ conjunto. Son equivalentes:
	\begin{enumerate}
		\item Una accion $G\times X \longrightarrow X$
		\item Un homomorfismo $\alpha : G\rightarrow \mathbb{S} (x)$
	\end{enumerate}\end{proposition}
\begin{proof}
pendiente
\end{proof}

\begin{example}
	\begin{enumerate}
		\item La accion trivial $G\times X\rightarrow X$ corresponde a
			\begin{align} 
				&G\longrightarrow \mathbb{S} (X)\nonumber\\
				&g\longmapsto Id_{x}\nonumber
			\end{align}
		\item La accion regular $G\times G\longrightarrow G$ corresponde al homomorfismo de Cayley (DUDA)$G$
	\end{enumerate}
\end{example}

\begin{definition}
	Sea $G\times X \longrightarrow X$ una accion de un grupo $G$ en $X\neq\emptyset$. Dos elementos $x,y\in X $ se dicen $G$-conjugados mediante esta accion si $\exists g\in G $ tal que $g.x=y$ (notacion $x\sim y$)

	Esto define una relacion de equivalencia en $X$ (Ejercici). Asi, tal relacion particiona a $X$ en clases de equivalencia

	Sea $x\in X $ entonces $G.x$ o $\mathcal{O}_{G}(x)$ es la clase de equivalencia de $x$ que se llamara $G$-Orbita de $x$ $$X=\bigcup_{x\in X }G.x $$
\end{definition}

\begin{remark}
	Si $G\times X \longrightarrow X$ es accion entonces cualquier subgrupo de $G$ actua en $X$ por restriccion. De este modo $G=\mathbb{S}_{n}$ actua naturalmente en $I_{n}$ 
	$$<\sigma>.j=\mathcal{O}_{\sigma}=\{\sigma^{k}:k\geq 0\}\quad \forall\sigma \in\mathbb{S}_{n} $$ 
\end{remark}

\begin{definition}[Accion Transitiva]
	Una accion se dice transitiva si posee una unica orbita es decir si $\exists x\in X $ tal que $X=G.x$
\end{definition}

\begin{definition}[G-Estabilizador]
	Sea $G\times X\longrightarrow X$ accion. Dado $x\in X$ el $G$-estabilizador de $x$ es $$G_{x}=\{g\in G: g.x=x\}$$
	$G_{x}$ es un subgrupo de $G$ , $\forall x\in X\quad\forall g,h\in G_{x}$ (No necesariamente normal)

	Si $\alpha :G\longrightarrow \mathbb{S}$ homomorfismo correspondiente a la accion dada entonces: $$Ker(\alpha)=\bigcap_{x_{i} X}G_{x}$$
\end{definition}

\begin{example}
	\begin{enumerate}
		\item $G\times X\longrightarrow G$ accion trivial $g.x=\{x\}$ entonces $G_{x}=G$
		\item $G\times G \longrightarrow G$ accion regular $g.x=gx$
			
			$G.x=G$ pues $y=(yx^{-1} )x=yx^{-1}.x$ ( Entonces es transitiva )

			$G_{x}=\{e\}$ pues $gx=x\iff g=e$
		\item $H\trianglelefteq G$, $G\times H \longrightarrow H$ por conjugacion $g.x=gxg^{-1} $

			$$G.x=\{gxg^{-1} : g\in G \}=Cl(X) \quad \text{(Clase de conjugacion de} X)$$ 
			$$G_{x}=\{g\in G: gxg^{-1} =x\}=C_{G}(x)\quad \text{(Centralizador de } x \text{ en } G)$$

			(ejercicios calcular estabilizador y centralizador de traslaciones para alfguna coclase)
		\item Sea $H\leq G$ con 
			$$G\times \quotient{G}{H}\longrightarrow \quotient{G}{H}$$ 
			dada por $g.aH=ga.H$ con $\quotient{G}{H}=\{aH: a\in G \}$

			Es accion transitiva porque $G.\quotient{G}{H}=\quotient{G}{H}$

			$G_{H}=\{g\in G : g.eH=ge.H=H\}=H$ ( DUDA )
	\end{enumerate}
\end{example}

\begin{proposition}
	Sea $G\times X\longrightarrow X$ una accion de $G$ en $X$, se tienen:
	\begin{enumerate} 
		\item $\forall x\in X,G_{g.x}=gG_{x}g^{-1} \quad\forall g\in G$
		\item $ \lvert G.x\rvert =[G:G_{x}] $ 
	\end{enumerate}
\end{proposition}

\begin{proof}
	Pendiente
\end{proof}

\begin{theorem}[Ecuacion de Clase]
	Sean $G$ grupo y $ G\times X\longrightarrow X $ una accion de $G$ en $ X\neq \emptyset$ $ \exists  $ famlia $ \{G_{i}\}_{i\in I } $ de sugrupos propios de $G$ tales que:
	$$\lvert X \rvert =\lvert X^{G} \rvert + \sum_{n=1}^{\mathbb{N}} $$
	donde $ X^{G} = \{x\in X: g.x=x\quad\forall g\in G \} $ (B$ G $-invariante)
\end{theorem}
\begin{proof}
	pendiente	
\end{proof}

\begin{theorem}[Teorema de Cauchy]\label{12.2}
	Sea $G$ grupo de orden $ n $ y sea $ p>0 $ primo tal que $ p|n $ entonces $G$ tiene un elemento de orden $ p $
\end{theorem}

\begin{proof}
	Pendiente
\end{proof}

\section{Clase 13}

\subsection{Duda}
pagina 1 teo13 parte gris no entiendo
\begin{remark}
	Si tenemos $\lvert G \rvert =p$ con $p$ primo y tomamos $e\neq x\in G$ como $\lvert x \rvert |p$ y $p$ primo entonces $\lvert x \rvert =p$, luego $G=<x>$ y $G\equiv \mathbb{Z}_{p}$   
\end{remark}

\begin{proposition}
	Sea $p$ primo con $\lvert G \rvert =p^{2}$ entonces $G$ es abeliano 
\end{proposition}
\begin{proof}
	
\end{proof}

\begin{remark}
	Grupos abelianos de orden $p^{2}$ (No isomorfos entre si) $\mathbb{Z}_{p^{2}}$ y $\mathbb{Z}_{p}\bigoplus \mathbb{Z}_{p}$ 
\end{remark}

\begin{definition}
	Un grupo $G$ se dice un p-grupo, con $p$ primo si $\forall x\in G \ \exists n\in \mathbb{N} $ tal que $x^{p^{n}}=e$ 
	
	Es decir todo elemento de $G$ tiene orden una potencia de $p$ 
\end{definition}

\begin{remark}
	Un \textbf{p-grupo} finito tiene orden potencia de $p$  
\end{remark}
\begin{proof}
	\begin{enumerate}
		\item Sea $\lvert G \rvert =p_{1}^{n_{1}}\ldots p_{k}^{n_{k}}$ con $p_{i}$ primos.
		\item Entonces $p_{i}\big|\lvert G \rvert $ por lo tanto $\exists x\in G\quad \lvert x \rvert = p_{i}$ (Por \hyperref[12.2]{Teorema de Cauchy} )
		\item Luego $p_{i}=\lvert x \rvert =p^j $ (Esto ultimo por ser \textbf{p-grupo}) 
		\item Entonces $j=1$ y $p_{i}=p\quad \forall 1\leq i\leq k$ 
		\item por lo tanto $\lvert G \rvert =p^j$ con $j>k$  
	\end{enumerate}  
\end{proof}

\begin{remark}
	Si $\lvert G \rvert =p^{3}$ y $G$ no abeliano entonces $G\times \mathbb{Z}_{p}\times\cdots\times\mathbb{Z}_{p}$ (k-veces) es no abeliano y $\lvert G \rvert =p^{k+3}$ 
\end{remark}

\begin{proposition}
	Sea $G$ un \textbf{p-grupo} finito, no tivial entonces $Z(G)=\{e\}$ (no trivial)
\end{proposition}
\begin{proof}
	Copias
\end{proof}

\begin{definition}[Normalizador]
	Sea $H\leq G$, el normalizador de $H$ en $G$ es el subgrupo
	$$N_{G}(H)= \{g\in G: gHg^{-1}=H\}$$ 
\end{definition}

\begin{remark}
	$N_{G}(H)$ es el estabilizador de $H\in \{\text{subgrupos de } G\}$ con respecto a la accion de $G$ por conjugacion (DUDA)
	
	Version mia sea $H\leq G$ y $G$ actuando sobre $H$ por conjugacion entonces $N_{G}(H)$ es el estabilizador de $H$ en $G$ 
\end{remark}

\begin{remark}
	$H\trianglelefteq N_{G}(H)$. Ademas $N_{G}(H)$ es el mayor subgrupo de $G$ que contiene a $H$ como un subgrupo normal. 

	En particular $$H\trianglelefteq G\iff N_{G}(H)=G$$
\end{remark}

\begin{lemma}
	Sean $p$ primo y $G$ un \textbf{p-grupo} finito. Si $H<G$ entonces $H<N_{G}(H)$. Concluyendo que $H=N_{G}(H)$ entonces $H=G$    
\end{lemma}

\begin{definition}
	Un grupo se llama simple si no contiene subgrupos normales distintos de $\{e\}$ y $G$
\end{definition}

\begin{corollary}
	Si $\lvert G \rvert =p^{n}$ y $H\leq G$ con $[G:H]=p$ entonces $H\trianglelefteq G$. En particular el unico \textbf{p-subgrupo} finito simple es $\mathbb{Z}_{p}$  
\end{corollary}


\section{Clase 14}

\begin{corollary}
	$ [G:S]=p $ entonces $ S \trianglelefteq G $
\end{corollary}
\begin{proof}
	
\end{proof}

\begin{proposition}
	$ |G|=p^{n} $ con $ n\in \mathbb{N}_{0} $ entonces:
	\begin{enumerate}
		\item  $ G $ posee subrupos de orden $ p^{i} \quad \forall 0\leq i\leq n$
		\item Si $0\leq i\leq n-1$ y $S\leq G$ con $|S|=p^{i}$ entonces $\exists $ subrupo $T$ de orden $p^{i+1}$ tal que $S \trianglelefteq T$ 
	\end{enumerate}
\end{proposition}

\subsection{Teoremas de Sylow}

\begin{remark}
	En esta seccion $G$ es grupo finito y $p$ primo
\end{remark}

\begin{definition}[p-grupo de Sylow]
	Un $p$-subgrupo de Sylow de $G$ es un subgrupo $H$  tal que $\lvert H \rvert =p^{n}$ donde $\lvert G \rvert =p^{n}k$ con $(p,k)=1$ 
\end{definition}

\subsection{Primer Teorema de Sylow}
\begin{theorem}[Primer Teorema de Sylow] \label{14.1}
	Supongamos que $\lvert G \rvert =p^{n}k$ con $(p,k)=1$. Entonces $\forall 0\leq i\leq n$ tenemos que $G$ posee un subgrupo de orden $p^{i}$. En particular $G$ posee un $p$-Sylow 
\end{theorem}
\begin{proof}
	pendiente  
\end{proof}

\begin{theorem}[Segundo Teorema de Sylow]\label{14.2}
	Sea $G$ grupo finito y $p$ primo. Sean $S\leq  G$ tal que $\lvert S \rvert =p^{i}$ con $i\in N_{0}$ y $H$ un \textbf{p-subgrupo de Sylow} de $G$ entonces $\exists a\in G$ tal que $S\trianglelefteq aHa^{-1}$. 

	En particular $S$ es \textbf{p-sylow} si y solo si $S$ y $H$ son conjugados 
\end{theorem}
\begin{proof}
	
\end{proof}

\begin{corollary}[DUDA]
	Sea $H\leq  G$ \textbf{p-sylow} entonces $H$ es el unico \textbf{p-sylow} de $G$ si y solo si $H\trianglelefteq G$ 
\end{corollary}
\begin{proof}
	$(\Rightarrow )$ Supongamos que no es normal entonces $aHa^{-1} = J$ con $J\neq H$ entonces por ser $J$ es conjugado de $H$ es \textbf{p-sylow}. Absurdo por que $H$ era el unico \textbf{p-sylow} 

	$(\Leftarrow)$ Existencia no se , se que existe alguno pero no se si es H. Unicidad supongamos que no es unico entonces $\exists J$ \textbf{p-sylow} , entonces $J=aHa^{-1}$ pero $aHa^{-1}=H$ por ser $H$ normal 
\end{proof}

\begin{theorem}[Tercer Teorema Sylow]
	Sea $G$ grupo finito, $p$ primo y sea $n_{p}=\lvert \{\text{\textbf{p-sylow de } } G\} \rvert $ entonces $$n_{p}\bigg| \lvert G \rvert \quad \text{ y }\quad  n_{p}\equiv 1 (p)$$   
\end{theorem}


\section{Clase 16}
\begin{definition}
	Sea $R$ anillo. Un elemento $0\neq a\in R$ se \textbf{divisor de cero a izquierda (derehca)} si $\exists 0\neq b\in R$ tal que $ab=0$ (respectivamente $ba=0$ ) 
	
	Si $a$ es divisor de cero a izquierda y a derecha entonces se dice que $a$ es divisor de cero
\end{definition}

\begin{example}
	\begin{enumerate}
		\item En $\mathbb{Z}_{n}$ si $n$ no es primo tomamos $d|n$ entonces $\overline{d}$ es un divisor en $\mathbb{Z}_{n}$
		\item En $M_{n}(R)\quad n>1$ tomamos COPIAR MATRICES   
	\end{enumerate}
\end{example}

\begin{definition}
	Un anillo conmutativo con identidad $1\neq 0$ se dice de dominio integro (o de integridad) si no posee divisores de 0
\end{definition}

\begin{example}
	\begin{enumerate}
		\item $\mathbb{Z},\mathbb{R},\mathbb{Q},\mathbb{C} $ son dominios de integridad
		\item $\mathbb{Z}_{n}$ es dominio de integridad sii $n$ es primo observamos que $\mathbb{Z}_{n}=\quotient{\mathbb{Z} }{n\mathbb{Z} }$ 
		\item $\mathbb{Z}[x],\mathbb{Z} [x_{1},\ldots,x_{k}]$ etc , son dominios de integridad (Los polonomios con eoficientes en $\mathbb{Z} $ y variables $x_{i}$ ) 
	\end{enumerate}
\end{example}

\begin{definition}[Anillo inversible]
	Sea $R$ un anillo con identidad y sea $0\neq a\in R$ se dice que $a$ es:
	\begin{itemize}
		\item Inversible a izquierda $\iff\exists b\in R$ tal que $ba=1$    
		\item Inversible a derecha $\iff\exists b\in R$ tal que $ab=1$
		\item Inversible si lo es a derecha y a izquierda
	\end{itemize}

	Un anillo $D$ con $1\neq 0$ donde todo elemento es inversible se llama \textbf{anillo de division} 
\end{definition}

\begin{remark}
	Si $a\in R$ es inversible entonces el inverso a izquierda de $a$ coincide con su inverso a derecha y esta univocamente determiando por $a$ (Notacion: $a^{-1}$)  
\end{remark}

\begin{definition}
	El conjunto de los elementos inversibles en un anillo $R$ (con $1\neq 0$ ) se llama \textbf{grupo de unidades de R}. 

	(Notacion: $R^{X}$ o $R^{*}$ o $\mathcal{U}(R))$   )
\end{definition}

\begin{example}
	\begin{enumerate}
		\item 
	\end{enumerate}
\end{example}

\end{document}
